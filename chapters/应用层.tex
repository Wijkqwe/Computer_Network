
\chapter{应用层}

\section{网络应用模型}

\subsection{C/S模型}
客户/服务器模型。

\subsubsection{特点}
\begin{itemize}
    \item 客户、服务器地位不平等
    \item 客户之间不直接通信
    \item 服务器损害则影响全局
    \item 服务器负载大
\end{itemize}


\subsection{P2P模型}
\begin{itemize}
    \item 主机之间直接通信
    \item 主机之间地位平等
    \item 单个节点损害不影响全局
    \item 各节点可分摊负载
\end{itemize}


\section{DNS域名系统}
DNS:将域名解析为IP地址。

\subsection{域名}
www(三级域名).baidu(二级域名).com(顶级域名)

\begin{enumerate}
    \item 根域名
    \item 顶级域名:\begin{itemize}
        \item 国家顶级域名:cn、us、uk
        \item 通用顶级域名:com、net、org
        \item 基础结构域名/反向域名:arpa
    \end{itemize}
    \item 二级域名:\begin{itemize}
        \item 类别域名:ac、com、edu、net、org、gov
        \item 行政域名:js、bj,用于省、自治区、直辖市
    \end{itemize}
    \item 三级域名
    \item 四级域名
\end{enumerate}


\subsection{域名服务器}
\begin{enumerate}
    \item 根域名服务器:
    \item 顶级域名服务器:管理服务器注册的所有二级域名
    \item 权限域名服务器:负责一个区的域名服务器
    \item 本地域名服务器:主机发送DNS查询请求时,查询请求报文交给本地域名服务器;
\end{enumerate}


\subsection{域名解析过程}

\begin{itemize}
    \item 递归查询
    \item 迭代查询
\end{itemize}


\section{FTP}
\textbf{文件传送协议}。提供不同种类主机系统之间文件传输能力。
\begin{itemize}
    \item 基于C/S。
    \item 依照FTP,提供服务,进行文件传送的计算机是\textbf{FTP服务器}。
    \item 连接FTP服务器,依照FTP进行文件传送的计算机是\textbf{FTP客户端}。
\end{itemize}

\subsection{原理}

使用TCP实现可靠传输。TCP控制连接端口21;TCP数据连接端口20。
\begin{itemize}
    \item 控制连接始终保持
    \item 数据连接间隔保持
    \item 是否使用TCP20建立数据连接与传输模式相关\begin{itemize}
        \item 主动方式使用TCP20
        \item 被动方式由客户端与服务器协商决定(端口>1024)
    \end{itemize}
\end{itemize}

\subsubsection{登录}
FTP地址:用户名\&密码

\paragraph{匿名登录}


\subsection{传输模式}
\begin{enumerate}
    \item 文本模式:ASCII,以文本序列传输数据
    \item 二进制模式:Binary,以二进制序列传输数据
\end{enumerate}


\subsection{TFTP简单文件传送协议}


