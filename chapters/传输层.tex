
\chapter{传输层}

\section{端口}

\subsection{作用}

\begin{itemize}
    \item 通过\textbf{端口号}标识本机的一个进程\begin{itemize}
        \item 每台主机的端口号相互独立
        \item TCP、UDP端口号相互独立
    \end{itemize}
    \item TCP或UDP通过\textbf{Socket套接字=\{IP地址:端口号\}}唯一地标识网络上一台主机地一个应用进程
\end{itemize}


\subsection{分类}
\begin{itemize}
    \item 服务端使用\begin{itemize}
        \item 熟知端口号\(0\sim 1023\),通常只能用于被熟知的重要应用程序
        \item 登记端口号\(1024 \sim 49151\)
    \end{itemize}
    \item 客户端使用:短暂端口号\(49152 \sim 65535\)
\end{itemize}


\section{功能}
\begin{itemize}
    \item 实现端到端(进程到进程)的通信
    \item 复用和分用\begin{itemize}
        \item 复用(从上到下):发送数据时,同一主机上多个进程可以使用同一个传输层协议
        \item 分用(从下到上):传输层可以把数据正确交付到目的进程
    \end{itemize}
    \item 差错检测\begin{itemize}
        \item TCP:丢弃数据,通知发送方重传
        \item UDP:丢弃数据,不通知发送方
    \end{itemize}
    \item 向应用层提供服务:\begin{itemize}
        \item \textbf{面向连接的、可靠的}端到端传输服务TCP
        \item \textbf{无连接的、不可靠的}端到端传输服务UDP
    \end{itemize}
\end{itemize}

\subsection{连接}
\begin{itemize}
    \item 有连接:确认对方准备好接收数据
    \item 无连接:
\end{itemize}

\subsection{可靠}
\begin{itemize}
    \item 可靠:接收方使用确认机制,让发送方知道哪些数据被接收
    \item 不可靠:
\end{itemize}


\section{UDP}

\subsection{数据报}
\begin{itemize}
    \item 首部小,仅8B
    \item 每次传输一个完整报文,\textbf{不支持自动拆分重装},因此应用层报文长度不能超过UDP上限
    \item 不支持拥塞控制
    \item 不可靠,可靠性靠应用层控制
    \item 支持一对一、一对多
\end{itemize}

\subsubsection{格式}
首部8B:|16位源端口号|16位目的端口号|16位UDP长度|16位UDP校验和|

理论最大长度65535B。


\subsection{校验}
32bit数据+16bit校验和

数据以16bit为一组做加法(最高位进位\textbf{回卷})后取反得到\textbf{校验和}。

以16bit为一组做加法(最高位进位\textbf{回卷}),若无比特错误,则结果一定全1.

计算校验和之前添加伪首部;计算完成后去掉伪首部。

\paragraph{伪首部格式}
|源IP地址4B|目的IP地址4B|0(1B)|17(1B)|UDP长度2B|

\subsubsection{发送方}
\begin{enumerate}
    \item 计算校验和之前添加伪首部;
    \item 把伪首部、首部、数据部分以16bit为一组做加法(最高位进位\textbf{回卷})
    \item 加法结果按位取反,得到16bit校验位,填入UDP首部;
    \item 去掉伪首部,将UDP数据报交给网络层,封装为IP数据报。
\end{enumerate}

\subsubsection{接收方}
\begin{enumerate}
    \item 网络层向传输层递交UDP数据报
    \item 添加伪首部
    \item 把伪首部、首部、数据部分以16bit为一组做加法(最高位进位\textbf{回卷})
    \item 若全1,则无错,接收数据报,根据目的端口号向应用层提交报文;若不全1,则有错,丢弃数据报。
\end{enumerate}


\section{TCP}

\subsection{数据报}
\begin{itemize}
    \item 首部大,\(20\sim60\)B
    \item 支持报文自动拆分重装,可以传输长报文
    \item 支持拥塞控制
    \item 仅支持一对一
\end{itemize}

\subsubsection{格式}
首部\begin{itemize}
    \item 固定首部20B\begin{itemize}
        \item 源端口16bit
        \item 目的端口16bit
        \item 序号seq,32bit,标记数据部分第一个字节在原始字节流中的位置
        \item 确认号ack/ack\_seq,32bit,用于反馈,表示序号在该确认号之前的所有字节已正确接收
        \item 数据偏移4bit,TCP首部长度,以*4B为单位,因此首部最长15*4B=60B
        \item 保留6bit,没用,通常全0
        \item \begin{itemize}
            \item URG,=1时紧急指针有效,应插队传输
            \item \textbf{ACK(1bit)}\footnote{仅第一次握手时ACK=0,其余TCP报文段ACK=1}
            \begin{itemize}
                \item =0时,ack\_seq无效
                \item =1时,ack\_seq有效,可称为ACK段
            \end{itemize}
            \item PSH,=1时表示希望接收方尽快回复
            \item RST,=1时表示出现严重差错,需释放连接
            \item \textbf{SYN}\footnote{仅第一次握手、第二次握手SYN=1,其余全0},=1时可称为SYN段,表示是连接请求或连接接受报文
            \item \textbf{FIN}\footnote{仅第一次挥手、第三次挥手FIN=1,其余为0},=1时可称为FIN段,表示此报文发送方数据发送完毕,要求释放传输连接
        \end{itemize}
        \item 窗口rwnd/rcvwnd,16bit,表示接收窗口大小,从本报文首部ack\_seq算起,接收方还能接收多少数据;是实现\textbf{流量控制}的关键
        \item 校验和,原理与UDP类似
        \item 紧急指针
    \end{itemize}
    \item 选项(长度可变)\begin{itemize}
        \item 建立TCP时,在第一次握手,第二次握手的选项中协商\textbf{MSS(最大段长)}
        \item MSS表示在接下来的数据传输中,TCP数据部分最多携带多少数据
        \item 若MSS过大,在IP层会被切片
    \end{itemize}
    \item 填充,凑4B整数倍
\end{itemize}
TCP数据部分长度根据IP首部、TCP首部中的信息算出。


\subsection{过程}
\begin{enumerate}
    \item 建立连接(三次握手),对应3个TCP报文段
    \item 双向传输TCP段(全双工通信)
    \item 释放连接(四次挥手):
\end{enumerate}

每次TCP可传输多个报文;

TCP是面向字节流的,UDP是面向报文的










































































