
\chapter{例}

\section{数据链路层}

\subsubsection{GBN协议}
数据链路层采用GBN协议,发送方发送了编号为0-7的帧。当计计时器超时时,若发送方仅收到0,2,3确认帧,则发送方应重传的帧数为

\subparagraph{解}
连续ARQ协议中,接收方采用累积确认,收到3确认帧代表0,1,2,3已收到,未收到1代表确认帧返回过程丢失了,不代表1帧未到达接收方。故需重传4,5,6,7。


\subsubsection{以太网交换机}
以太网交换机进行转发决策时使用的PDU地址为

\subparagraph{解}
交换机工作在数据链路层,数据链路层使用物理地址进行转发,转发通常根据目的地址决定出端口。



\subsubsection{CSMA/CD}
在一个采用CSMA/CD协议的网络中,传输介质是一根完整电缆,传输速率为1Gbps,电缆中信号传播速度为200000km/s。若最小数据帧长度减少800bit,则最远两个站点之间距离至少需要:

\subparagraph{解}
若最短帧长减少,而数据传输速率不变,则需是冲突域最大距离变短来实现争用期减少。争用期为网络中收发的\textbf{往返时延}。设减少的最小距离s,则有\(2 * \dfrac{s}{2 * 10^8} = \dfrac{800}{10^9}\),得s=80。


