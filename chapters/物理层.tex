
\chapter{物理层}

实现相邻节点之间bit的传输。

\section{通信基础基本概念}

\subsection{信源}


\subsection{信宿}


\subsection{信号}
数据的载体。

\subsubsection{数字信号}
信号值是离散的。


\subsubsection{模拟信号}
信号值是连续的。


\subsection{信道}


\subsection{码元}
一个信号即一个码元,信号周期即码元宽度。

一个幅值即一种信号。


\subsubsection{码元携带比特}

若一个信号周期内可能出现4种信号,那么每个信号可以对应一个4进制数(2bit)。

若一个码元可能存在4种状态,则称其为4进制码元(一个码元携带2bit数据);

若一个码元可能存在8种状态,则称其为8进制码元(一个码元携带3bit数据);

若一个周期内可能存在k种信号,则一个码元可携带\(\log_2\)k bit数据。


\subsection{波特率}
每秒传输几个码元。单位:码元/秒,或波特Baud。


\subsection{比特率}
每秒传输几个比特。单位:bps。


\section{信道的极限容量}

\subsection{噪声}


\subsection{奈奎斯特定理}
对一个理想低通信道(没有噪声,带宽有限),极限波特率=2W,单位:波特(W为信道的频率带宽,单位Hz)。

极限比特率=2W\(\log_2\)k bit b/s。


\subsection{香农定理}
对一个有噪声的、带宽有限的信道,

极限比特率=W\(\log_2\)(1 + S/N)(W为信道的频率带宽,单位Hz;S/N信噪比,无单位)


\subsection{信噪比}
无单位记法: = S/N = \(\dfrac{\text{信号的功率}}{\text{噪声的功率}}\)(无单位)

分贝记法: = 10\(\log_{10}\)S/N(单位分贝)


\section{编码与调制}

\subsection{概念}

\paragraph{编码}
二进制数据到数字信号

\paragraph{解码}
数字信号到二进制数据

\paragraph{调制}
二进制数据到模拟信号

\paragraph{解调}
模拟信号到二进制数据


\subsection{常用编码方式}
\begin{itemize}
    \item NRZ不归零编码:低0高1,中不变
    \item RZ归零编码:低0高1,中归零
    \item NRZI反向非归零编码:跳0不跳1,看起点,中不变
    \item 曼彻斯特编码:跳0反跳1,看中间,中必变;\textbf{以太网}默认使用;
    \item 差分曼彻斯特编码:跳0不跳1,看起点,中必变
\end{itemize}

\subsubsection{曼彻斯特两种标准}
\begin{enumerate}
    \item 上0下1,IEEE标准,408常用
    \item 下0上1
\end{enumerate}


\subsubsection{各种编码特点}


\subsection{常用调制方式}
\begin{itemize}
    \item AM调幅(辐移键控ASK):\(y = \begin{cases}
        0 * \sin2x,\ 0 \\ 
        1 * \sin2x,\ 1
    \end{cases}\)
    \item FM调频(频移键控FSK):\(y = \begin{cases}
        \sin1x,\ 0 \\ 
        \sin2x,\ 1
    \end{cases}\)
    \item PM调相(相移键控PSK):\(y = \begin{cases}
        \sin(x + 0),\ 0 \\ 
        \sin(x + \pi),\ 1
    \end{cases}\)
    \item QAM正交幅度调制:AM、PM结合起来的叠加信号,m种幅值、n种相位,mn种信号。
\end{itemize}

\subsubsection{QAM调制方案}
\begin{itemize}
    \item QAM-16:16种信号,4bit
    \item QAM-32:32种信号,5bit
    \item QAM-64:64种信号,6bit
    \item QAM-128:128种信号,7bit
\end{itemize}


\section{传输介质}

\subsection{导向型}

\subsubsection{双绞线}
有屏蔽层:屏蔽双绞线STP;没有屏蔽层:非屏蔽双绞线UTP。

抗干扰能力:较好。绞合、屏蔽层提供。

应用:近年局域网、早期电话线。


\subsubsection{同轴电缆}
内导体(传输信号) + 外导体屏蔽层(抗干扰)


应用:早期局域网、早期有线电视


\subsubsection{光纤}
纤芯(高折射率) + 包层(低折射率),利用光的全反射。

单模光纤:

多模光纤:


\subsection{以太网对有线传输介质的命名规则}
速度 + Base + 介质信息
\begin{itemize}
    \item 10Base5:10Mbps,同轴电缆,最远传输距离500m
    \item 10Base2:10Mbps,同轴电缆,最远传输距离200m
    \item 10BaseF*:10Mbps,光纤,*可以是其他信息
    \item 10BaseT*:10Mbps,双绞线,*可以是其他信息
\end{itemize}


\subsection{非导向型}
无线传输介质。

本质都是电磁波,\(C = \lambda F\),C光速,\(\lambda\)波长,F频率。

\subsubsection{无线电波}

如:手机信号、WiFi


\subsubsection{微波通信}
保密性差。

如:微星通信


\subsubsection{其他}
红外线通信,激光通信。

信号指向性强。


\section{物理层接口特性}

\subsection{机械特性}


\subsection{电气特性}


\subsection{功能特性}


\subsection{过程特性(规程特性)}


\section{物理层设备}

\subsection{中继器}
若传输距离太长,数字信号会失真。

中继器只有两个端口,一个接收信号,失真信号整形再生,转发至另一端口。

仅支持半双工通信。两个端口对应两个网段。


\subsection{集线器}
本质是多端口中继器。将其中一个端口收到的信号,整形再生后转发至其余端口。

各端口连接的结点不可同时发送数据,会冲突。

N个端口对应N个网段,各网段属于同一\textbf{冲突域}。


\subsubsection{冲突域}
若两主机同时发送数据会导致冲突,则两个主机位于同一冲突域中。

同一冲突域的主机发送数据前需信道争用

\begin{itemize}
    \item 集线器不能隔离冲突域。
    \item 以太网交换机可以隔离冲突域
\end{itemize}


\subsection{特性}
\begin{itemize}
    \item 中继器、集线器不能无限串联
    \item 集线器 连接的网络,物理上星形拓扑,逻辑上总线型拓扑
    \item 集线器 连接的各网段共享带宽
\end{itemize}

\subsubsection{5-4-3原则}
使用集线器或中继器连接10Base5网段时,最多只能串联5个网段,使用4台集线器或中继器,只有3个网段可以挂接计算机。



