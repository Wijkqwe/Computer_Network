
\chapter{物理层}

实现相邻节点之间bit的传输。

\section{通信基础基本概念}

\subsection{信源}


\subsection{信宿}


\subsection{信号}
数据的载体。

\subsubsection{数字信号}
信号值是离散的。


\subsubsection{模拟信号}
信号值是连续的。


\subsection{信道}


\subsection{码元}
一个信号即一个码元,信号周期即码元宽度。

一个幅值即一种信号。


\subsubsection{码元携带比特}

若一个信号周期内可能出现4种信号,那么每个信号可以对应一个4进制数(2bit)。

若一个码元可能存在4种状态,则称其为4进制码元(一个码元携带2bit数据);

若一个码元可能存在8种状态,则称其为8进制码元(一个码元携带3bit数据);

若一个周期内可能存在k种信号,则一个码元可携带\(\log_2\)k bit数据。


\subsection{波特率}
每秒传输几个码元。单位:码元/秒,或波特Baud。


\subsection{比特率}
每秒传输几个比特。单位:bps。


\section{信道的极限容量}

\subsection{噪声}


\subsection{奈奎斯特定理}
对一个理想低通信道(没有噪声,带宽有限),极限波特率=2W,单位:波特(W为信道的频率带宽,单位Hz)。

极限比特率=2W\(\log_2\)k bit b/s。


\subsection{香农定理}
对一个有噪声的、带宽有限的信道,

极限比特率=W\(\log_2\)(1 + S/N)(W为信道的频率带宽,单位Hz;S/N信噪比,无单位)


\subsection{信噪比}
无单位记法: = S/N = \(\dfrac{\text{信号的功率}}{\text{噪声的功率}}\)(无单位)

分贝记法: = 10\(\log_{10}\)S/N(单位分贝)






