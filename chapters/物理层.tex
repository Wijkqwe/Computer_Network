
\chapter{物理层}

实现相邻节点之间bit的传输。

\section{通信基础基本概念}

\subsection{信源}


\subsection{信宿}


\subsection{信号}
数据的载体。

\subsubsection{数字信号}
信号值是离散的。


\subsubsection{模拟信号}
信号值是连续的。


\subsection{信道}


\subsection{码元}
一个信号即一个码元,信号周期即码元宽度。

若一个信号周期内可能出现4种信号,那么每个信号可以对应一个4进制数(2bit)。

若一个码元可能存在4种状态,则称其为4进制码元(一个码元携带2bit数据);

若一个码元可能存在8种状态,则称其为8进制码元(一个码元携带3bit数据);



\subsection{速率}


\subsection{波特}







