
\chapter{网络层}

为传输层提供服务,将传输层数据封装成IP数据报。网络中路由器根据IP数据报首部源IP地址、目的IP地址进行分组转发,实现“主机到主机”的传输。

\section{功能}

\subsection{异构网络互联}

\paragraph{异构}
每个网络拓扑结构不同,物理层、链路层实现不同、主机类型不同。


\paragraph{重要设备}
路由器。TCP/IP文献中,路由器又称网关。


\subsection{路由与转发}

\subsubsection{路由}
各路由器之间相互配合,规划IP数据报的最佳转发路径。

各路由器需要允许路由协议,最终生成各自的路由表。


\subsubsection{转发}
一台路由器根据自己的\textbf{转发表},将IP数据报从合适的接口转发出去。

转发表 = 精简路由表。数据结构更简洁,便于快速检索。


\subsection{拥塞控制}

\subsubsection{拥塞}

\subparagraph{原因}
出现过量分组,超负荷,性能下降。


\subparagraph{现象}
分组数增加,吞吐率下降。


\subsubsection{开环控制(静态)}
部署网络时,提前设计好预防拥塞的方法,一旦开始运行便不再修改。


\subsubsection{闭环控制(动态)}
动态监测网络状态,及时发现拥塞,将拥塞信息传给路由器。相关路由器及时调整路由表。


\section{IP数据报(IP分组)}

\subsection{格式}
\begin{itemize}
    \item 首部(最小20B,最大\(2^4 * 4B = 60B\)\begin{itemize}
        \item 固定部分(20B):|版本4bit\footnote{区分网络层使用的IP协议版本(v4、v6)}|首部长度4bit\footnote{0\(\sim\)15,以*4B为单位}|区分服务8bit\footnote{一般用不到}|总长度16bit\footnote{0\(\sim\)65535,以*1B为单位,涵盖首部和数据部分}|
        
        |标识16bit\footnote{由IP数据报源主机生成,一般为自增序列}|标志3bit\footnote{最低位MF,MF=1,后面还有分片;MF=0,后面没有分片。次低位DF,DF=1,不允许被切片;DF=0,允许被切片}|片偏移13bit\footnote{数据部分在被分片前的位置,以*8B为单位}|
        
        |生存时间8bit\footnote{数据报在网络中可通过的路由器数的最大值TTL。初值由源主机设置,每经过一个路由器TTL-=1;若减到0,则丢弃,并向源主机发送ICMP报文}|协议8bit\footnote{若为TCP,则设为6;若为UDP,则设为17}|首部检验和16bit\footnote{每个路由器仅校验首部,若该字段设为全0,表示不用校验;校验和计算方法与UDP相同}|
        
        |源地址32bit发送方IP地址|
        
        |目的地址32bit接收方IP地址|
        
        \item 可变部分(0\(\sim\)40B):可选字段(长度可变)|填充(凑足4B的整数倍)
    \end{itemize}
    \item 数据部分:最小0B,理论最长=65535-20=65515B
\end{itemize}


\subsection{分片问题}
一个数据链路层帧能承载的最大数据量称为\textbf{最大传送单元MTU}。若一个IP数据报的总长度超出了下一段链路的MTU,则需要分片。

每个分片都可被独立转发,即都包含首部。
\begin{itemize}
    \item 分片可能在\textbf{源主机或任一路由器}中发生
    \item 只有目的主机会对分片进行重组
    \item 各分片可能乱序到达
    \item 由于首部的片偏移字段以*8B为单位,故除最后一个分片外,其余分片数据部分为8B的整数倍
\end{itemize}


\subsection{无连接不可靠}
如果是面向连接的,则应有用于建立连接的字段,但是没有;如果提供可靠的服务,则至少应有序号和校验和两个字段,但是IP分组头中也没有(IP首部中只是首部校验和)。


\section{IPv4}

\subsubsection{各协议之间关系}

IP协议是互联网的核心。

ARP协议用于查询同一网络中,<IP, MAC>之间的映射关系。

ICMP协议用于网络层各实体之间相互通知异常事件。

IGMP协议用于实现IP组播。


\subsection{IP分类方案}
\begin{itemize}
    \item 单播地址\begin{itemize}
        \item A类(1-126):|0|网络号1-7|主机号8-31|
        \item B类(128-191):|1|0|网络号2-15|主机号16-31|
        \item C类(192-223):|1|1|0|网络号3-23|主机号24-31|
    \end{itemize}
    \item 多播地址:D类(224-239):|1|1|1|0|多播地址|
    \item 保留使用:E类(240-255):|1|1|1|1|保留使用|
\end{itemize}
若一个网络中,主机号占nbit,则该网络中最多支持\(2^n - 2\)台主机或路由器。

\subsubsection{特殊IP地址}
\begin{center}
    \begin{tabular}{c|c}
        \hline
        网络号 & 主机号 \\
        \hline
        Y & 全0 \\ 
        \hline
        Y & 全1 \\ 
        \hline
        0 & Y \\ 
        \hline
        全0 & 全0 \\
        \hline
        全1 & 全1 \\ 
        \hline
        127 & 非全0且非全1 \\ 
        \hline
    \end{tabular}
\end{center}


\subsection{子网划分}
设某IP地址段,主机号占n bit,可将前k bit作为子网号,剩下n-k bit作为主机号,则可划分出\(2^k\)个子网。每个子网包含IP地址块大小相同。

\begin{itemize}
    \item 划分前:IP地址两级结构<网络号,主机号>
    \item 划分后:IP地址三级结构<网络号,子网号,主机号>
\end{itemize}

每个子网地址中,主机号不能全0或全1.


\subsection{子网掩码}
子网掩码、IP地址逐位与得到<网络号,子网号>(合称为\textbf{网络前缀})。只有网络前缀相同的IP地址,才能归属同一网络(子网)。

若一个网络内部进行子网划分,则网络中每台主机、路由器接口都需要配置IP地址、默认网关、子网掩码。

若一台路由器支持子网划分技术,则转发表中,应包含<目的网络号,子网掩码,转发接口>


\subsection{默认子网掩码}
若一个传统网络(A/B/C类)内没有进行子网划分,则将对应此网络的\textbf{转发表项}设为默认子网掩码。


\subsection{默认路由}
默认路由(默认转发表项)设置:<目的网络号全0,子网掩码全0>

在路由转发表中,若所有表项都不匹配,则转发默认路由。


\subsection{主机发送IP数据报的过程}
\begin{enumerate}
    \item 判断目的主机和本机是否处于同一网络\begin{enumerate}
        \item 检测本机IP地址和目的IP地址网络前缀是否相同。相同则同一网络;否则非同一网络。
    \end{enumerate}
    \item 将IP数据报封装成MAC帧并发送到链路上\begin{itemize}
        \item 若属于同一网络,则通过ARP协议找到目的主机的MAC地址,将IP数据报封装成帧,将帧发给\textbf{目的主机}。
        \item 若不属于同一网络,通过ARP协议找到默认网关的MAC地址,将IP数据报封装成帧,将帧发给\textbf{默认网关}。
    \end{itemize}
\end{enumerate}


\subsection{路由器转发IP数据报的过程}
\begin{enumerate}
    \item 路由器某个接口收到一个IP数据报
    \item 对数据报首部进行校验,从中找到\textbf{目的IP地址}
    \item 查转发表\begin{itemize}
        \item 转发表表项包含<目的网络号,子网掩码,转发接口>
        \item 检查目的IP地址与每个表项是否能匹配(目的IP地址、子网掩码逐位与,匹配表项中的目的网络号)
        \item 至少默认路由可以匹配成功
    \end{itemize}
    \item 转发\begin{itemize}
        \item 根据3的结果,将IP数据报从匹配的接口中转发出去
        \item 若匹配的转发接口与入口相同,则不再发送
    \end{itemize}
\end{enumerate}


\subsection{CIDR无分类编址}

\paragraph{原因}
传统IP分类方式资源分配不灵活,利用率低,有限资源很快耗尽。

32bit IP地址:网络前缀(可变长)|主机号


\subsubsection{IP地址结构}
IP地址=<网络前缀,主机号>,其中网络前缀不定长。

记法:xxx.xxx.xxx.xxx/k,网络前缀k bit,主机号32-k bit



\subsubsection{定长子网划分}
CIDR地址块中,主机号前kbit作为定长子网号,可划分\(2^k\)个子网。
每个子网可分配的最大IP地址个数为\(2^{16 - k} - 2\)(全0全1不可私用)


\subsubsection{变长子网划分}
CIDR地址块中,子网号长度不固定,每个子网含IP地址块大小不同。

类似于从根构造哈夫曼树。\begin{itemize}
    \item 原始CIDR地址快为根
    \item 每个分支节点必须同时拥有左右孩子,左0右1或相反
    \item 每个叶子节点对应一个子网,根据\textbf{从根节点到叶节点的路径}分析子网对应\textbf{IP地址范围}
    \item 整棵树高度不超过h-1,最小子网也保留2bit主机号
\end{itemize}


\subsection{路由聚合}
又称\textbf{构成超网}。对一个路由转发表,若几条路由表项转发接口相同,部分网络前缀相同,则可将这几条路由表项聚合为一条。

可减少路由表大小。

可能引入额外的无效地址。

\subsubsection{最长路由匹配原则}
一个IP地址在转发表中可能匹配多个表项,应用最长路由匹配原则。


\subsection{DHCP动态主机配置协议}
是\textbf{应用层协议},基于UDP:客户端UDP端口号68,服务器端UDP端口号67

\subsubsection{作用}
给刚接入网络的主机动态分配IP地址,配置默认网关、子网掩码。


\subsubsection{使用C/S客户端服务器模型}
\begin{itemize}
    \item 客户:新接入网络的主机
    \item 服务器:负责分配IP地址的主机,管理一系列IP地址池\begin{itemize}
        \item 家庭网络中,通常由家庭路由器兼容DHCP服务器
        \item 在一个大型网络中,可存在多个DHCP服务器
    \end{itemize}
\end{itemize}


\subsubsection{过程}
\begin{enumerate}
    \item 客户 -> 服务器:DHCP发现报文
    \item 服务器 -> 客户:DHCP提供报文
    \item 客户 -> 服务器:DHCP请求报文
    \item 服务器 -> 客户:DHCP确认报文
\end{enumerate}


\subsection{NAT网络地址转换}

\subsubsection{私有IP地址(内网IP)}
\[\begin{cases}
    10.0.0.0 \sim 10.255.255.255 \\
    172.16.0.0 \sim 172.31.255.255 \\
    192.168.0.0 \sim 192.168.255.255
\end{cases}\]
\begin{itemize}
    \item 只能分配给局域网内部节点
    \item 可复用,只要求局域网内唯一
    \item 局域网内部可自由分配
\end{itemize}


\subsubsection{全球IP地址(外网IP)}
通常由ISP提供,全球唯一。

是局域网与外界通信时使用的IP地址


\subsubsection{NAT路由器}

\paragraph{作用}
转发IP数据报时,进行内网IP、外网IP的相互转换。

\paragraph{NAT表}
记录地址转换关系<内网IP:端口号\(\leftrightarrow\)外网IP:端口号>

包含传输层功能(端口号是传输层概念)

一个IP数据报\(\begin{cases}
    \text{从内网到外网,会更改源IP地址、源端口号} \\ 
    \text{从外网到内网,会更改目的IP地址、目的端口号}
\end{cases}\)


\subsection{ARP地址解析协议}

\paragraph{作用}
查询同一网络中<IP地址,MAC地址>之间映射关系。


\paragraph{ARP表(ARP缓存)}
\begin{itemize}
    \item 记录<IP地址,MAC地址>
    \item 需定期更新ARP表项
    \item 每台主机、路由器有自己的ARP表
\end{itemize}

\subsubsection{过程}

\begin{enumerate}
    \item ARP请求分组\begin{itemize}
        \item 内容:谁?(IP地址X,MAC地址Y)找谁?(目的IP地址Z)
        \item ARP请求分组封装进\textbf{MAC帧}(帧目的地址全1,源地址Y)-\textbf{广播帧}
    \end{itemize}
    \item ARP响应分组\begin{itemize}
        \item 内容:IP地址Z,MAC地址V
        \item ARP响应分组封装进MAC帧(帧目的地址Y,源地址V)-\textbf{单播帧}
    \end{itemize}
\end{enumerate}


\section{ICMP网际控制报文协议}
属于网络层,ICMP报文封装在IP数据报中。

让主机或路由器相互报告网络中差错和异常情况。

\subsection{差错报告报文}
\begin{itemize}
    \item 终点不可达\(\begin{cases}
        \text{路由器->发送方:目的IP地址不可达} \\ 
        \text{目的主机->发送方:目的端口号不存在,没有对应进程}
    \end{cases}\)
    \item 时间超过\(\begin{cases}
        \text{路由器->发送方:IP数据报到这里TTL=0} \\ 
        \text{目的主机->发送方:IP数据报被切片,规定时间内未到齐,已全部丢弃}
    \end{cases}\)
    \item 参数问题-->发送方:IP数据报首部参数不合法,或首部校验出错
    \item 改变路由(重定向)-路由器->发送方:对这个网络,下次使用另一路由器,路径更短
    \item 源点抑制:是当路由器或主机由于\textbf{拥塞}而丢弃数据报时,就向源点发送源点抑制报文,使源点知道应当把数据报的发送速率放慢。
\end{itemize}


\subsection{询问报文}
\begin{itemize}
    \item 回送请求
    \item 回送回答
    \item 时间戳请求
    \item 时间戳回答
\end{itemize}


\subsection{不必反馈差错报告报文的情况}
\begin{itemize}
    \item 若携带ICMP差错报告报文的IP数据报出错,不再反馈
    \item 若IP数据报被分片,无论多少分片出错,只反馈一次
    \item 若IP数据报目的地址为多播,不反馈
    \item 若源地址特殊地址,不反馈
\end{itemize}


\subsection{典型应用}
\begin{itemize}
    \item ping:基于\textbf{回送请求报文、回送回答报文}实现功能
    \item tracert:基于\textbf{时间超过报文}实现
\end{itemize}


\section{IPv6地址}
128位。

\subsection{冒号十六进制记法}
共128位,每16位为一段,记录为16进制,段间冒号分隔


\subsection{压缩记法}
\begin{enumerate}
    \item 取出每段前导0
    \item 双冒号::替代连续出现的多个0,一个地址只能出现一次双冒号
\end{enumerate}


\subsection{资源分配}
\begin{itemize}
    \item 支持CIDR
    \item 支持即插即用(IP自动配置)
    \item 可以不使用DHCP,也支持使用DHCP统一管理
\end{itemize}


\subsection{分类}
\begin{center}
    \begin{tabular}{c|c}
        \hline
        地址类型 & 二进制前缀 \\
        \hline
        未指明 & 0...0(128位),记为::/128 \\ 
        \hline
        环回 & 0...01(128位),记为::1/128 \\
        \hline
        多播 & 11111111(8位),记为FF00::/8 \\
        \hline
        本地链路单播 & 1111111010(10位),记为FE80::/10 \\
        \hline
        全球单播 & 以上4种以外 \\
        \hline
    \end{tabular}
\end{center}

3中目的地址\begin{enumerate}
    \item 单播
    \item 多播:数据报发送到一组计算机的每一台
    \item 任播:终点是一组计算机,但数据报只交付其中一台计算机,通常是距离最近的计算机。没有固定前缀
\end{enumerate}


\section{路由算法}
路由选择算法。本质是图的最短路径问题。

根据\textbf{能否随网络的通信量或拓扑自适应地进行调整变化},分为两类:

\subsection{静态路由算法}
实现简单开销小,不具备自适应能力。

由网络管理员手动配置每条路由。


\subsection{动态路由算法}
实现复杂开销大,具备自适应能力。

路由器动态调整自身路由表。

\subsubsection{距离向量路由算法}
RIP协议使用。

不必关系完整的网络拓扑结构。

本质是Bellman-Ford算法在路由领域的应用。


\subsubsection{链路状态路由算法}
OSPF协议使用。

要求每台路由器节点都了解完整的网络拓扑结构。

只要一台路由器知道完整结构,可用\textbf{Dijkstra}找到最优路径。本质是Dijkstra在路由领域的应用。


\subsubsection{路径-向量路由算法}
BGP协议使用。考纲未明确要求。


\section{路由协议}

\subsection{RIP}
基于距离向量路由算法,网络数越多,距离向量越大。

\subsubsection{位置}
属于应用层,使用UDP传输数据(端口520),RIP报文位于UDP数据部分。
\begin{enumerate}
    \item Request报文:请求邻居路由器发送路由表,用于启动或查询
    \item Response报文:响应请求或定时发送给邻居路由器,携带完整或部分路由表(每个Response报文最多携带25各路由表项信息,若路由表太大,则拆分为多个报文)
\end{enumerate}


\subsubsection{定义路径长度}
\begin{enumerate}
    \item 使用\textbf{跳数(或距离)}衡量到达目的网络的距离(路由器到直连网络的距离为1,每经过一个路由器距离加1)
    \item 认为好的路由是跳数最少的
    \item 允许一条合法路径距离不能超过15。距离=16表示网络不可达,因此只适用于小型AS
\end{enumerate}


\subsubsection{定义路由表(距离向量)格式}
\begin{enumerate}
    \item 每个路由器维护自己的路由表,路由表项3个关键字段<目的网络N,距离d,下一跳路由器地址X>
    \item 每个路由器维护从它自身到其他每个目的网络的距离记录,即\textbf{距离向量},包含在路由表中
\end{enumerate}


\subsubsection{之间如何交换信息}
\begin{itemize}
    \item 仅和直接相邻的路由器交换信息
    \item 交换的信息是路由器的全部信息,即路由表(距离向量)
    \item 通常按固定时间间隔交换信息\begin{itemize}
        \item 可引入触发更新机制,路由器发现网络拓扑发生变化时,立即通告变换后的信息,不必等待间隔
    \end{itemize}
\end{itemize}


\subsubsection{过程}


\subsubsection{优劣}


\subsection{OSPF}
基于链路状态路由算法。

\subsubsection{位置}
属于网络层。使用IP协议提供的服务,IP首部协议字段=89。

OSPF中PDU常称为\textbf{OSPF分组}或\textbf{OSPF数据报}。定义了5中分组类型


\subsubsection{之间如何交换信息}
\begin{itemize}
    \item 说什么:与自己直连的所有链路状态\begin{itemize}
        \item 数据结构 图的顶点
        \item 带权有向边
    \end{itemize}
    \item 怎么说:洪泛法\begin{itemize}
        \item 洪泛信息不可回流转发
        \item 收到重复信息,不转发
        \item 一传十,十传百
    \end{itemize}
    \item 什么时候:路由器检测到直连链路或节点发生变化时,立即洪泛
\end{itemize}


\subsubsection{特点}
\begin{itemize}
    \item 允许对每条路由设置不同代价,对不同类型业务计算不同路由\begin{itemize}
        \item OSPF\textbf{默认基于带宽}计算链路代价metric=参考带宽/接口带宽(接口带宽越高,链路代价越小)(参考带宽默认100Mbps)
    \end{itemize}
    \item 若到同一网络有多条相同代价的路径,可将通信量分配给这几条路径
    \item OSPF分组有鉴别功能,保证仅在可信赖的路由器之间交换链路状态信息
    \item 支持变长子网划分和CIDR
    \item 每个链路状态信息带一个32位的序号,序号越大状态越新
\end{itemize}


\subsection{OSPF工作原理}

\subsubsection{生成路由表}
\begin{enumerate}
    \item 构建LSDB\begin{itemize}
        \item 路由器使用洪泛法将探测到的\textbf{链路状态信息}发送给其他路由器
        \item 各台路由器根据其他路由器发来的\textbf{链路状态信息}构建\textbf{LSDB链路状态数据库}(本质是图的邻接表)
    \end{itemize}
    \item 运行Dijkstra算法,基于LSDB,计算到每一个网络的最短路径
    \item 生成路由表,根据Dijkstra算法的结果,结构为<目的网络,下一条,到目的网络的距离>
\end{enumerate}


\subsubsection{区域划分}
将AS划分为一个主干区域,多个非主干区域

\paragraph{作用}
\begin{itemize}
    \item 洪泛范围局限在每个区域内部,降低洪泛压力
    \item 一台普通路由器只需关注所在区域的网络拓扑,LSDB变小
\end{itemize}


\subsubsection{几类路由器}

\paragraph{自治系统路由器ASBR}
主干区域内至少一台,与其他自治系统相连

\paragraph{区域边界路由器ABR}
每个非主干区域至少一台,与主干区域相连

\paragraph{区域内部路由器}
非边界路由器,每个区域内有多台


\subsection{OSPF分组}
定义5种类型
{\footnotesize\begin{center}
    \begin{tabular}{|c|c|c|c|}
        \hline
        Type值 & 英文缩写 & 中文译名 & 作用 \\
        \hline
        1 & Hello Packet & 问候分组 & 建立维持邻居关系(每10s发一次,40s超时) \\ 
        \hline
        2 & DD Packet & 数据库描述分组 & 邻居建立时,发送LSDB摘要(LSA的头信息) \\ 
        \hline
        3 & LSR Packet & 链路状态请求分组 & 向邻居请求缺失的LSA\footnote{LSA:链路状态通告,OSPF定义的一种数据结构}(指明头信息) \\ 
        \hline
        4 & LSU Packet & 链路状态更新分组 & 向邻居传送具体的LSA(可能引发全网洪泛) \\ 
        \hline
        5 & LSAck Packet & 链路状态确认分组 & 收到邻居的LSU后,向邻居确认收到哪些LSA(头信息) \\ 
        \hline
    \end{tabular}
\end{center}}


\subsection{BGP}


\subsection{分层次}
将世界网络划分为多个相互独立的自治系统AS。AS管理单位有权决定使用哪种\textbf{内部路由(网关)协议IGP}(RIP、OSPF)。

每个AS至少一台\textbf{自治系统边界路由器}与其他系统相连。

各边界路由器之间使用统一的\textbf{外部路由(网关)协议EGP}(BGP-4)。

自治系统之间:域间路由选择

自治系统内部:域内路由选择

每个自治系统有全球唯一的AS编号。


\section{IP多播(组播)}
属于多播组的设备被分配一个\textbf{组播组IP地址}(有共同需求的主机的相同标识)。

组播地址范围为:224.0.0.0 \(\sim\) 239.255.255.255(D类地址)。一个D类地址表示一个组播组,只能用于\textbf{目的地址},源地址只能是\textbf{单播地址}。

\subsection{硬件组播}


\subsection{IGMP网际组管理协议}


\subsection{组播路由选择协议}


\section{移动IP}
移动节点以固定的网络IP地址,实现跨越不同网段的漫游功能,保证基于网络IP的网络权限在漫游过程中不会改变。

\subsection{移动节点}


\subsection{移动代理(本地代理)}


\subsection{永久地址(归属地址/主地址)}


\subsection{外部代理(外地代理)}


\subsection{转交地址(辅地址)}


\subsection{通信过程}


\section{网络层设备}

\subsection{路由器}




























































