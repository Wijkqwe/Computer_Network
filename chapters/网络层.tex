
\chapter{网络层}

为传输层提供服务,将传输层数据封装成IP数据报。网络中路由器根据IP数据报首部源IP地址、目的IP地址进行分组转发,实现“主机到主机”的传输。

\section{功能}

\subsection{异构网络互联}

\paragraph{异构}
每个网络拓扑结构不同,物理层、链路层实现不同、主机类型不同。


\paragraph{重要设备}
路由器。TCP/IP文献中,路由器又称网关。


\subsection{路由与转发}

\subsubsection{路由}
各路由器之间相互配合,规划IP数据报的最佳转发路径。

各路由器需要允许路由协议,最终生成各自的路由表。


\subsubsection{转发}
一台路由器根据自己的\textbf{转发表},将IP数据报从合适的接口转发出去。

转发表 = 精简路由表。数据结构更简洁,便于快速检索。


\subsection{拥塞控制}

\subsubsection{拥塞}

\subparagraph{原因}
出现过量分组,超负荷,性能下降。


\subparagraph{现象}
分组数增加,吞吐率下降。


\subsubsection{开环控制(静态)}
部署网络时,提前设计好预防拥塞的方法,一旦开始运行便不再修改。


\subsubsection{闭环控制(动态)}
动态监测网络状态,及时发现拥塞,将拥塞信息传给路由器。相关路由器及时调整路由表。


\section{IPv4}

\subsubsection{各协议之间关系}

IP协议是互联网的核心。

ARP协议用于查询同一网络中,<IP, MAC>之间的映射关系。

ICMP协议用于网络层各实体之间相互通知异常事件。

IGMP协议用于实现IP组播。


\section{IP数据报(IP分组)}

\subsection{格式}
\begin{itemize}
    \item 首部(最小20B,最大\(2^4 * 4B = 60B\)\begin{itemize}
        \item 固定部分(20B):|版本4bit\footnote{区分网络层使用的IP协议版本(v4、v6)}|首部长度4bit\footnote{0\(\sim\)15,以*4B为单位}|区分服务8bit\footnote{一般用不到}|总长度16bit\footnote{0\(\sim\)65535,以*1B为单位,涵盖首部和数据部分}|
        
        |标识16bit\footnote{由IP数据报源主机生成,一般为自增序列}|标志3bit\footnote{最低位MF,MF=1,后面还有分片;MF=0,后面没有分片。次低位DF,DF=1,不允许被切片;DF=0,允许被切片}|片偏移13bit\footnote{数据部分在被分片前的位置,以*8B为单位}|
        
        |生存时间8bit\footnote{数据报在网络中可通过的路由器数的最大值TTL。初值由源主机设置,每经过一个路由器TTL-=1;若减到0,则丢弃,并向源主机发送ICMP报文}|协议8bit\footnote{若为TCP,则设为6;若为UDP,则设为17}|首部检验和16bit\footnote{每个路由器仅校验首部,若该字段设为全0,表示不用校验;校验和计算方法与UDP相同}|
        
        |源地址32bit发送方IP地址|
        
        |目的地址32bit接收方IP地址|
        
        \item 可变部分(0\(\sim\)40B):可选字段(长度可变)|填充(凑足4B的整数倍)
    \end{itemize}
    \item 数据部分:最小0B,理论最长=65535-20=65515B
\end{itemize}


\subsection{分片问题}
一个数据链路层帧能承载的最大数据量称为\textbf{最大传送单元MTU}。若一个IP数据报的总长度超出了下一段链路的MTU,则需要分片。

每个分片都可被独立转发,即都包含首部。
\begin{itemize}
    \item 分片可能在\textbf{源主机或任一路由器}中发生
    \item 只有目的主机会对分片进行重组
    \item 各分片可能乱序到达
    \item 由于首部的片偏移字段以*8B为单位,故除最后一个分片外,其余分片数据部分为8B的整数倍
\end{itemize}








