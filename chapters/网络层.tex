
\chapter{网络层}

为传输层提供服务,将传输层数据封装成IP数据报。网络中路由器根据IP数据报首部源IP地址、目的IP地址进行分组转发,实现“主机到主机”的传输。

\section{功能}

\subsection{异构网络互联}

\paragraph{异构}
每个网络拓扑结构不同,物理层、链路层实现不同、主机类型不同。


\paragraph{重要设备}
路由器。TCP/IP文献中,路由器又称网关。


\subsection{路由与转发}

\subsubsection{路由}
各路由器之间相互配合,规划IP数据报的最佳转发路径。

各路由器需要允许路由协议,最终生成各自的路由表。


\subsubsection{转发}
一台路由器根据自己的\textbf{转发表},将IP数据报从合适的接口转发出去。

转发表 = 精简路由表。数据结构更简洁,便于快速检索。


\subsection{拥塞控制}

\subsubsection{拥塞}

\subparagraph{原因}
出现过量分组,超负荷,性能下降。


\subparagraph{现象}
分组数增加,吞吐率下降。


\subsubsection{开环控制(静态)}
部署网络时,提前设计好预防拥塞的方法,一旦开始运行便不再修改。


\subsubsection{闭环控制(动态)}
动态监测网络状态,及时发现拥塞,将拥塞信息传给路由器。相关路由器及时调整路由表。


\section{IPv4}

\subsubsection{各协议之间关系}

IP协议是互联网的核心。

ARP协议用于查询同一网络中,<IP, MAC>之间的映射关系。

ICMP协议用于网络层各实体之间相互通知异常事件。

IGMP协议用于实现IP组播。


\subsection{IP数据报(IP分组)的格式}
\begin{itemize}
    \item 首部(最小20B,最大\(2^4 * 4B = 60B\)\begin{itemize}
        \item 固定部分(20B):|版本4bit\footnote{区分网络层使用的IP协议版本(v4、v6)}|首部长度4bit\footnote{0\(\sim\)15,以*4B为单位}|区分服务8bit\footnote{一般用不到}|总长度16bit\footnote{0\(\sim\)65535,以*1B为单位,涵盖首部和数据部分}|标识|标志|片偏移|生存时间|协议|首部检验和|源地址|目的地址
        \item 可变部分(0\(\sim\)40B):可选字段(长度可变)|填充(凑足4B的整数倍)
    \end{itemize}
    \item 数据部分:最小0B,理论最长=65535-20=65515B
\end{itemize}





