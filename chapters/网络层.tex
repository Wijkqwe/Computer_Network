
\chapter{网络层}

为传输层提供服务,将传输层数据封装成IP数据报。网络中路由器根据IP数据报首部源IP地址、目的IP地址进行分组转发,实现“主机到主机”的传输。

\section{功能}

\subsection{异构网络互联}

\paragraph{异构}
每个网络拓扑结构不同,物理层、链路层实现不同、主机类型不同。


\paragraph{重要设备}
路由器。TCP/IP文献中,路由器又称网关。


\subsection{路由与转发}

\subsubsection{路由}
各路由器之间相互配合,规划IP数据报的最佳转发路径。

各路由器需要允许路由协议,最终生成各自的路由表。


\subsubsection{转发}
一台路由器根据自己的\textbf{转发表},将IP数据报从合适的接口转发出去。

转发表 = 精简路由表。数据结构更简洁,便于快速检索。


\subsection{拥塞控制}

\subsubsection{拥塞}

\subparagraph{原因}
出现过量分组,超负荷,性能下降。


\subparagraph{现象}
分组数增加,吞吐率下降。


\subsubsection{开环控制(静态)}
部署网络时,提前设计好预防拥塞的方法,一旦开始运行便不再修改。


\subsubsection{闭环控制(动态)}
动态监测网络状态,及时发现拥塞,将拥塞信息传给路由器。相关路由器及时调整路由表。


\section{IP数据报(IP分组)}

\subsection{格式}
\begin{itemize}
    \item 首部(最小20B,最大\(2^4 * 4B = 60B\)\begin{itemize}
        \item 固定部分(20B):|版本4bit\footnote{区分网络层使用的IP协议版本(v4、v6)}|首部长度4bit\footnote{0\(\sim\)15,以*4B为单位}|区分服务8bit\footnote{一般用不到}|总长度16bit\footnote{0\(\sim\)65535,以*1B为单位,涵盖首部和数据部分}|
        
        |标识16bit\footnote{由IP数据报源主机生成,一般为自增序列}|标志3bit\footnote{最低位MF,MF=1,后面还有分片;MF=0,后面没有分片。次低位DF,DF=1,不允许被切片;DF=0,允许被切片}|片偏移13bit\footnote{数据部分在被分片前的位置,以*8B为单位}|
        
        |生存时间8bit\footnote{数据报在网络中可通过的路由器数的最大值TTL。初值由源主机设置,每经过一个路由器TTL-=1;若减到0,则丢弃,并向源主机发送ICMP报文}|协议8bit\footnote{若为TCP,则设为6;若为UDP,则设为17}|首部检验和16bit\footnote{每个路由器仅校验首部,若该字段设为全0,表示不用校验;校验和计算方法与UDP相同}|
        
        |源地址32bit发送方IP地址|
        
        |目的地址32bit接收方IP地址|
        
        \item 可变部分(0\(\sim\)40B):可选字段(长度可变)|填充(凑足4B的整数倍)
    \end{itemize}
    \item 数据部分:最小0B,理论最长=65535-20=65515B
\end{itemize}


\subsection{分片问题}
一个数据链路层帧能承载的最大数据量称为\textbf{最大传送单元MTU}。若一个IP数据报的总长度超出了下一段链路的MTU,则需要分片。

每个分片都可被独立转发,即都包含首部。
\begin{itemize}
    \item 分片可能在\textbf{源主机或任一路由器}中发生
    \item 只有目的主机会对分片进行重组
    \item 各分片可能乱序到达
    \item 由于首部的片偏移字段以*8B为单位,故除最后一个分片外,其余分片数据部分为8B的整数倍
\end{itemize}


\section{IPv4}

\subsubsection{各协议之间关系}

IP协议是互联网的核心。

ARP协议用于查询同一网络中,<IP, MAC>之间的映射关系。

ICMP协议用于网络层各实体之间相互通知异常事件。

IGMP协议用于实现IP组播。


\subsection{IP分类方案}
\begin{itemize}
    \item 单播地址\begin{itemize}
        \item A类(1-126):|0|网络号1-7|主机号8-31|
        \item B类(128-191):|1|0|网络号2-15|主机号16-31|
        \item C类(192-223):|1|1|0|网络号3-23|主机号24-31|
    \end{itemize}
    \item 多播地址:D类(224-239):|1|1|1|0|多播地址|
    \item 保留使用:E类(240-255):|1|1|1|1|保留使用|
\end{itemize}
若一个网络中,主机号占nbit,则该网络中最多支持\(2^n - 2\)台主机或路由器。

\subsubsection{特殊IP地址}
\begin{center}
    \begin{tabular}{c|c}
        \hline
        网络号 & 主机号 \\
        \hline
        Y & 全0 \\ 
        \hline
        Y & 全1 \\ 
        \hline
        0 & Y \\ 
        \hline
        全0 & 全0 \\
        \hline
        全1 & 全1 \\ 
        \hline
        127 & 非全0且非全1 \\ 
        \hline
    \end{tabular}
\end{center}


\subsection{子网划分}
设某IP地址段,主机号占n bit,可将前k bit作为子网号,剩下n-k bit作为主机号,则可划分出\(2^k\)个子网。每个子网包含IP地址块大小相同。

\begin{itemize}
    \item 划分前:IP地址两级结构<网络号,主机号>
    \item 划分后:IP地址三级结构<网络号,子网号,主机号>
\end{itemize}

每个子网地址中,主机号不能全0或全1.


\subsection{子网掩码}
子网掩码、IP地址逐位与得到<网络号,子网号>(合称为\textbf{网络前缀})。只有网络前缀相同的IP地址,才能归属同一网络(子网)。

若一个网络内部进行子网划分,则网络中每台主机、路由器接口都需要配置IP地址、默认网关、子网掩码。

若一台路由器支持子网划分技术,则转发表中,应包含<目的网络号,子网掩码,转发接口>


\subsection{默认子网掩码}
若一个传统网络(A/B/C类)内没有进行子网划分,则将对应此网络的\textbf{转发表项}设为默认子网掩码。


\subsection{默认路由}
默认路由(默认转发表项)设置:<目的网络号全0,子网掩码全0>

在路由转发表中,若所有表项都不匹配,则转发默认路由。


\subsection{主机发送IP数据报的过程}
\begin{enumerate}
    \item 判断目的主机和本机是否处于同一网络\begin{enumerate}
        \item 检测本机IP地址和目的IP地址网络前缀是否相同。相同则同一网络;否则非同一网络。
    \end{enumerate}
    \item 将IP数据报封装成MAC帧并发送到链路上\begin{itemize}
        \item 若属于同一网络,则通过ARP协议找到目的主机的MAC地址,将IP数据报封装成帧,将帧发给\textbf{目的主机}。
        \item 若不属于同一网络,通过ARP协议找到默认网关的MAC地址,将IP数据报封装成帧,将帧发给\textbf{默认网关}。
    \end{itemize}
\end{enumerate}


\subsection{路由器转发IP数据报的过程}
\begin{enumerate}
    \item 路由器某个接口收到一个IP数据报
    \item 对数据报首部进行校验,从中找到\textbf{目的IP地址}
    \item 查转发表\begin{itemize}
        \item 转发表表项包含<目的网络号,子网掩码,转发接口>
        \item 检查目的IP地址与每个表项是否能匹配(目的IP地址、子网掩码逐位与,匹配表项中的目的网络号)
        \item 至少默认路由可以匹配成功
    \end{itemize}
    \item 转发\begin{itemize}
        \item 根据3的结果,将IP数据报从匹配的接口中转发出去
        \item 若匹配的转发接口与入口相同,则不再发送
    \end{itemize}
\end{enumerate}


\subsection{CIDR无分类编址}

\paragraph{原因}
传统IP分类方式资源分配不灵活,利用率低,有限资源很快耗尽。

32bit IP地址:网络前缀(可变长)|主机号


\subsubsection{IP地址结构}
IP地址=<网络前缀,主机号>,其中网络前缀不定长。

记法:xxx.xxx.xxx.xxx/k,网络前缀k bit,主机号32-k bit



\subsubsection{定长子网划分}
CIDR地址块中,主机号前kbit作为定长子网号,可划分\(2^k\)个子网。
每个子网可分配的最大IP地址个数为\(2^{16 - k} - 2\)(全0全1不可私用)


\subsubsection{变长子网划分}
CIDR地址块中,子网号长度不固定,每个子网含IP地址块大小不同。

类似于从根构造哈夫曼树。\begin{itemize}
    \item 原始CIDR地址快为根
    \item 每个分支节点必须同时拥有左右孩子,左0右1或相反
    \item 每个叶子节点对应一个子网,根据\textbf{从根节点到叶节点的路径}分析子网对应\textbf{IP地址范围}
    \item 整棵树高度不超过h-1,最小子网也保留2bit主机号
\end{itemize}


\subsection{路由聚合}
又称\textbf{构成超网}。对一个路由转发表,若几条路由表项转发接口相同,部分网络前缀相同,则可将这几条路由表项聚合为一条。

可减少路由表大小。

可能引入额外的无效地址。

\subsubsection{最长路由匹配原则}
一个IP地址在转发表中可能匹配多个表项,应用最长路由匹配原则。


\subsection{DHCP动态主机配置协议}
是\textbf{应用层协议},基于UDP:客户端UDP端口号68,服务器端UDP端口号67

\subsubsection{作用}
给刚接入网络的主机动态分配IP地址,配置默认网关、子网掩码。


\subsubsection{使用C/S客户端服务器模型}
\begin{itemize}
    \item 客户:新接入网络的主机
    \item 服务器:负责分配IP地址的主机,管理一系列IP地址池\begin{itemize}
        \item 家庭网络中,通常由家庭路由器兼容DHCP服务器
        \item 在一个大型网络中,可存在多个DHCP服务器
    \end{itemize}
\end{itemize}


\subsubsection{过程}
\begin{enumerate}
    \item 客户 -> 服务器:DHCP发现报文
    \item 服务器 -> 客户:DHCP提供报文
    \item 客户 -> 服务器:DHCP请求报文
    \item 服务器 -> 客户:DHCP确认报文
\end{enumerate}


\subsection{NAT网络地址转换}

\subsubsection{私有IP地址(内网IP)}






