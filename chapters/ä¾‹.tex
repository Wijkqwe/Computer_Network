
\chapter{例}

\section{体系结构}

\subsubsection{存储转发机制2010\_34}
在采用“存储-转发”方式的分组交换网络中,所有速率的数据传输速率为100Mbit/s,分组大小为1000B,其中分组头大小20B,若主机H1向主机H2发送一个大小为980000B的文件,则在不考虑分组拆装时间和传播延时的情况下,从H1发送开始到H2接收完成为止,需要的时间至少为

\textbf{解:}
需要\(\dfrac{980000}{1000 - 20} = 1000\)个分组,当\(t = \dfrac{1M * 8}{100Mbps} = 80ms\)时,H1发送完毕。 \\
由于传输延时,H1发送完成时,还有两个未到达,最短路径经过2个交换机转发到达目的地。每次转发时间\(t_0 = \dfrac{1K * 8}{100Mbps} = 0.08ms\)。 \\
综上,至少\(80.16\)ms。


\section{物理层}

\subsubsection{波特率}
若某通信链路的数据传输速率为2400bps,采用4相位调制,则该链路的波特率是

\textbf{解:}
有4种相位,则一个码元需要由\(\log_24=2\)个bit表示,则波特率=比特率 / 2 = 1200波特。


\section{数据链路层}

\subsubsection{GBN协议}
数据链路层采用GBN协议,发送方发送了编号为0-7的帧。当计计时器超时时,若发送方仅收到0,2,3确认帧,则发送方应重传的帧数为

\textbf{解:}
连续ARQ协议中,接收方采用累积确认,收到3确认帧代表0,1,2,3已收到,未收到1代表确认帧返回过程丢失了,不代表1帧未到达接收方。故需重传4,5,6,7。


\subsubsection{以太网交换机}
以太网交换机进行转发决策时使用的PDU地址为

\textbf{解:}
交换机工作在数据链路层,数据链路层使用物理地址进行转发,转发通常根据目的地址决定出端口。


\subsubsection{CSMA/CD}
在一个采用CSMA/CD协议的网络中,传输介质是一根完整电缆,传输速率为1Gbps,电缆中信号传播速度为200000km/s。若最小数据帧长度减少800bit,则最远两个站点之间距离至少需要:

\textbf{解:}
若最短帧长减少,而数据传输速率不变,则需是冲突域最大距离变短来实现争用期减少。争用期为网络中收发的\textbf{往返时延}。设减少的最小距离s,则有\(2 * \dfrac{s}{2 * 10^8} = \dfrac{800}{10^9}\),得s=80。


\subsubsection{SR}
数据链路层采用选择重传协议(SR)传输数据,发送方已发送了0~3号数据帧,现已收到1号帧的确认,而0、2号帧依次超时,则此时需要重传的帧数是?

\textbf{解:}
选择重传协议中,接收方逐个地确认正确接收的分组,不管接收到的分组是否有序,只要正确接收就发送选择ACK分组进行确认。因此选择重传协议中的ACK分组不再具有累积确认的作用。这点要特别注意与GBN协议的区别。此题中只收到1号帧的确认,0、2号帧超时,由于对于1号帧的确认不具累积确认的作用,因此发送方认为接收方没有收到0、2号帧,于是重传这两帧。


\subsubsection{GBN}
两台主机之间的数据链路层采用后退N帧协议(GBN)传输数据,数据传输速率为
16kbps, 单向传播时延为270ms,数据帧长度范围是128\(\sim\)512字节,接收方总是以与数据帧等长的帧进行确认。为使信道利用率达到最高,帧序号的比特数至少为

\textbf{解:}
考查GBN协议。本题主要求解的是从发送一个帧到接收到这个帧的确认为止的时间内最多可以发送多少数据帧。要尽可能多发帧,应以短的数据帧计算,因此首先计算出发送一帧的时间:\(128 * 8/(16 * 10^3)=64\)ms ;发送一帧到收到确认为止的总时间:\(64+270*2+64=668\)ms;这段时间总共可以发送668/64=10.4(帧),发送这么多帧至少需要用4位比特进行编号。


\subsubsection{以太网交换机直通延迟}
对于100Mbps的以太网交换机,当输出端口无排队,以直通交换方式转发一个以太网帧(不包括前导码)时,引入的转发延迟至少是

\textbf{解:}
直通交换检查包头(前14B),获取包目的地址,由于不考虑前导码,故只检查目的地址(6B),故最短\(0.48us\)


\section{网络层}

\subsubsection{子网划分}
某网络IP地址空间为192.168.5.0/24,采用定长子网划分,子网掩码为255.255.255.248,则该网络中最大子网数,每个子网内最大可分配地址个数为

\textbf{解:}
网络号为前24位。子网掩码25-32位的\(248_2 = 11111000\),故前5位用于子网号,后3位用于主机号。未使用CIDR,故不能使用全0全1。因此该网络空间的最大子网个数为\(2^5 = 32\),每个子网内的最大可分配地址个数为\(2^3 - 2 = 6\)


\subsubsection{最大主机数}
在子网192.168.4.0/30中,能接收目的地址为192.168.4.3的IP分组的最大主机数是?

\textbf{解:}
首先分析192.168.4.0/30这个网络。 \\
主机号占两位,地址范围192.168.4.0/30 \(\sim\) 192.168.4.3/30,即可以容纳(4-2=2)个主机。主机位为全1时,即192.168.4.3,是广播地址,因此网内所有主机都能收到,故2个。


\section{传输层}

\subsubsection{TCP流量控制拥塞控制}
主机甲乙之间已建立一个TCP连接,TCP最大段长度为1000B,若主机甲当前拥塞窗口为4000B,在甲向乙连续发送两个最大段后,成功收到乙发送的对第一个段的确认段,确认段中通告的接收窗口大小为2000B,则此甲可以向乙发送的最大字节数是

\textbf{解:}
发送窗口为MIN\{4000, 2000\}=2000B,发送方还没收到第二个段确认,故还可发送2000 - 1000 = 1000B


\subsubsection{2011-47/以太网帧/ARP/IP数据报}
某主机的MAC地址为00-15-C5-C1-5E-28,IP地址为10.2.128.100(私有地址)。题47-a 图是网络拓扑,题47-b图是该主机进行Web请求的1个以太网数据帧前80个字节的十六进制及ASCII码内容。 请参考图中的数据回答以下问题。 图略。\begin{itemize}
    \item (1)Web服务器的IP地址是什么?该主机的默认网关的MAC地址是什么? 
    \item (2)该主机在构造题47-b图的数据帧时,使用什么协议确定目的MAC地址?封装该协议请求报文的以太网帧的目的MAC地址是什么? 
    \item (3)假设 HTTP/1.1 协议以持续的非流水线方式工作,一次请求-响应时间为 RTT,rfc.html 页面引用了 5 个JPEG小图像,则从发出题47-b图中的Web请求开始到浏览器收到全部内容为止,需要多少个RTT? 
    \item (4)该帧所封装的IP分组经过路由器R转发时,需修改IP分组头中的哪些字段? 
\end{itemize}
注:以太网数据帧结构和IP分组头结构分别如题47-c图、题47-d图所示

\textbf{解:}
\begin{enumerate}
    \item 以太网帧头部6+6+2=14字节,IP数据报首部目的IP地址字段前有4*4=16字节,从以太网数据帧第一字节开始数14+16=30字节,得目的IP地址40 aa 62 20(十六进制),转换为十进制得64.170.98.32。以太网帧的前六字节 00-21-27-21-51-ee 是目的 MAC 地址,本题中即为主机的默认网关10.2.128.1端口的MAC地址。 
    \item ARP 协议解决IP地址到MAC地址的映射问题。主机的ARP进程在本以太网以广播的形式发送 ARP 请求分组,在以太网上广播时,以太网帧的目的地址为全 1,即 FF-FF -FF-FF-FF-FF。
    \item HTTP/1.1 协议以持续的非流水线方式工作时,服务器在发送响应后仍然在一段时间内保持这段连接,客户机在收到前一个响应后才能发送下一个请求。第一个RTT用于请求web页面,客户机收到第一个请求的响应后(还有五个请求未发送),每访问一次对象就用去一个RTT。故共1+5=6个RTT后浏览器收到全部内容。
    \item 私有地址和Internet上的主机通信时,须有NAT路由器进行网络地址转换,把IP数据报的源IP地址(本题为私有地址10.2.128.100)转换为NAT路由器的一个全球IP地址(本题为101.12.123.15)。因此,源IP地址字段0a 02 80 64变为65 0c 7b 0f。IP数据报每经过一个路由器,生存时间TTL值就减1,并重新计算首部校验和。若IP分组的长度超过输出链路的MTU,则总长度字段、标志字段、片偏移字段也要发生变化。
\end{enumerate}

