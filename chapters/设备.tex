
\chapter{设备}


\subsection{冲突域}
连接到同一物理介质上的所有节点的集合。节点之间存在介质争用现象(冲突)。

OSI中,物理层设备不能划分冲突域;2、3层设备可以。


\subsection{广播域}
接收同样广播消息的节点的集合。该集合内任一节点发出一个广播帧,能接收帧的节点在该广播域内。

OSI中,1、2层设备位于同一广播域;3层设备路由器可划分广播域。LAN(局域网)特质使用路由器划分的网络。


\subsection{网络风暴}
\begin{itemize}
    \item 物理层设备中继器和集线器既不隔离冲突域也不隔离广播域
    \item 网桥可隔离冲突域,但不隔离广播域
    \item 网络层的路由器既隔离冲突域,也隔离广播域
    \item VLAN 即虚拟局域网也可隔离广播域
\end{itemize}
对于不隔离广播域的设备,他们互连的不同网络都属于同一个广播域,因此扩大了广播域的范围,更容易产生网络风暴。
