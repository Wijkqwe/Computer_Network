
\chapter{数据链路层}

\section{功能}

使用物理层提供的\textbf{比特传输}服务;

为网络层提供服务,将网络层的\textbf{IP数据报(分组)}封装称帧,发送给下一结点;


\paragraph{物理链路}
传输介质(0层)+ 物理层(1层)实现相邻结点之间物理链路

\paragraph{逻辑链路}
数据链路层基于物理链路,实现相邻结点之间逻辑上无差错的“数据链路(逻辑链路)”


\subsection{封装成帧(组帧)}

\subsubsection{帧定界}
如何让接收方确定帧的界限


\subsubsection{透明传输}
接收方能够去除帧定界的附加信息


\subsubsection{字符计数法}
每个帧的开头用一个\textbf{定长计数字段}表示帧长。

帧长 = 定长计数字段长度 + 数据部分长度

缺点:任一计数字段出错,会导致后续所有帧无法定界。


\subsubsection{字节填充法}
使用控制字符SOH表示帧开始,EOT表示结束。

若帧的数据部分包含特殊字符(控制字符SOH、EOT,转义字符ESC),则发送方在特殊字符前添加转义字符ESC;接收方做逆处理。


\subsubsection{零比特填充法}
使用特殊比特串`01111110`表示帧开始/结束。

发送方对帧数据部分进行处理,每遇到连续的5个1,填充一个0;接收方做逆处理,每遇到连续的5个1,删除一个0.

HDLC协议、PPP协议使用。


\subsubsection{违规编码法}
违规编码基于曼彻斯特编码IEEE,上跳0下跳1,看中间,中必变。

需要物理层配合。

若周期中间不跳变,则违规。


\subsection{差错控制}

\paragraph{目标}
发现并解决一个帧内部的“位错”。


\subsubsection{检错编码}
接收方\textbf{发现比特错}后丢弃帧,通知发送方重传帧。

\paragraph{奇偶校验码}
整个校验码(有效信息位+校验位)中1的个数。
\begin{itemize}
    \item 奇校验码:1的个数为奇数;
    \item 偶校验码:1的个数为偶数;
\end{itemize}
\subparagraph{偶校验硬件实现}
各信息位进行\textbf{异或}运算,结果为偶校验位。将偶校验的信息位、校验位异或,结果为0则没有错误。

仅能检测处奇数位错误,无纠错能力。


\paragraph{CRC码(循环冗余校验)}
发送接收方约定一个“除数”,K信息位、R校验位作为“被除数”,添加校验位后保证除法余数为0。

\subparagraph{构造}
设生成多项式\(G(x)\),R位数为\(G(x)\)最高次幂,二进制码为多项式系数。信息位左移R位,对二进制码进行模2除法,得到余数即除数,除数有R + 1位。

\subparagraph{纠错}
\begin{itemize}
    \item 可检测所有奇数个错误
    \item 可检测所有双比特错误
    \item 可检测所有小于等于校验位长度的连续错误
    \item 若选择合适多项式,且\(2^R >= K + R + 1\),则可纠正单比特错
\end{itemize}


\subsubsection{纠错编码}

\paragraph{海明校验码}
信息位分组进行偶校验->多个校验位->标注出错位置
\subparagraph{位数}
n位信息位,k位校验位,则\(2^k >= n + k + 1\)。校验位\(P_i\)放到海明位\(2^{i - 1}\)上,信息位按顺序放到其余位置。
\subparagraph{海明位}
\(H_{n + k} ...H_2 H_1\)
\subparagraph{校验位}
\(P_k...P_2P_1\)。以7位海明码为例,\(P_1 = H_3 \oplus H_5 \oplus H_7 = D_1 \oplus D_2 \oplus D_4\)(3低位1,5低位1,6低位0,7低位1)
\subparagraph{格式转换}
海明位\(H_1H_2...H_{n + k}\),校验位\(P_1P_2...P_1\)
\subparagraph{纠错}
\(S_1 = P_1 \oplus D_1 \oplus D_2 \oplus D_4\),0则无错
\subparagraph{纠错能力}
1位
\subparagraph{检错能力}
2位
\subparagraph{全校验位}
\(P_{\text{全}} = S_k...S_1\)\begin{itemize}
    \item 为0且偶校验成功\(\Rightarrow\)无错
    \item 不为0且偶校验失败\(\Rightarrow\)1位错,纠正即可
    \item 不为0且偶校验成功\(\Rightarrow\)2位错,需重传
\end{itemize}


\subsection{可靠传输}



\subsection{流量控制}


\subsection{滑动窗口机制}
可靠传输与流量控制基于滑动窗口机制实现。

接收方通过确认机制控制发送方窗口滑动


\subsection{停止等待协议S-W}
不属于滑动窗口协议。
\begin{itemize}
    \item 滑动窗口:发送窗口\(W_T = 1\),接收窗口\(W_R = 1\)
    \item 确认机制:确认帧ACK\_i,接收方收到i号帧,没有检出错,返回确认帧ACK\_i
    \item 重传机制:超时重传
    \item 帧编号:仅需1bit给帧编号,需要\(W_T + W_R <= 2^n\);用于区别重复帧
\end{itemize}

\subsubsection{异常情况}
\begin{itemize}
    \item 数据帧丢失:超时重传,重置计时器,计时器超时前收到确认帧
    \item 确认帧丢失:超时重传,接收方收到重复帧,丢弃帧并返回重复帧的ACK
    \item 数据帧出错:丢弃帧,等发送方超时重传
\end{itemize}


\subsection{后退N帧协议GBN}
\begin{itemize}
    \item 滑动窗口:发送窗口\(W_T > 1\),接收窗口\(W_R = 1\)
    \item 确认机制:确认帧ACK\_i,接收方收到i号帧,没有检出错,返回确认帧ACK\_i;连续收到多个数据帧,仅传最后一个ACK\_i
    \item 重传机制:超时重传;若未收到ACK\_i,重传i及之后所有帧
    \item 帧编号:至少需nbit给帧编号,需要\(W_T + W_R <= 2^n\);
\end{itemize}

\subsubsection{异常情况}
\begin{itemize}
    \item 数据帧丢失:
    \item 确认帧丢失:
    \item 数据帧出错:
\end{itemize}


\subsection{选择重传协议SR}
\begin{itemize}
    \item 滑动窗口:发送窗口\(W_T > 1\),接收窗口\(W_R > 1\)
    \item 确认机制:\begin{itemize}
        \item 确认帧ACK\_i,接收方收到i号帧,没有检出错,返回确认帧ACK\_i;不支持累计确认
        \item 否认帧:NAK\_i,接收方收到i帧,但出错,丢弃帧并返回NAK\_i
    \end{itemize}
    \item 重传机制:\begin{itemize}
        \item 超时重传;
        \item 请求重传,发送方收到NAK\_i,则重传i
    \end{itemize}
    \item 帧编号:至少需nbit给帧编号,需要\(W_T + W_R <= 2^n\)且\(W_R <= W_T\);
\end{itemize}

\subsubsection{异常情况}
\begin{itemize}
    \item 数据帧丢失:
    \item 确认帧丢失:
    \item 数据帧出错:
\end{itemize}


\subsection{三种协议}
\begin{itemize}
    \item 滑动窗口协议:S-W不属于滑动窗口协议;GBN、SR属于滑动窗口协议。
    \item ARQ协议(自动重传请求协议):包括S-W、GBN、SR。\begin{itemize}
        \item 连续ARQ协议:只包括GBN、SR。
    \end{itemize}
\end{itemize}

\subsubsection{信道利用率}
\(T_D\)数据传输时延,\(RTT\)两倍的单向传输时延,\(T_A\)确认帧传输时延,发送窗口大小\(N\)。
\begin{itemize}
    \item S-W:理想情况下\footnote{没有帧丢失、比特错误等异常情况}\(U = \dfrac{T_D}{T_D + RTT + T_A}\),若\(T_A\)极短,则通常可忽略不计。
    \item GBN、SR:\(U = \dfrac{NT_D}{T_D + RTT + T_A}\);当\(NT_D >= T_D + RTT + T_A\),则U为1(不能大于1);常结合考察帧编号\footnote{\(W_T + W_R <= 2^n\)}。
    \begin{itemize}
        \item 相同nbit给帧编号,GBN\(W_R\)更小,\(W_T\)更大,故信道利用率也更大。
    \end{itemize}
\end{itemize}


\section{介质访问控制MAC}

\subsection{信道划分}

\subsubsection{时分复用TDM}
时间分为等长TDM帧,每个帧分为等长m个时隙,将m个时隙分给m对用户(节点)使用。

缺点:每个节点最多分配到信道总带宽的\(\dfrac{1}{m}\);若某节点暂不发送数据,时隙闲置,信道利用率低。


\subsubsection{统计时分复用STDM}
又称\textbf{异步时分复用}。在TDM基础上,动态分配时隙。

优点:


\subsubsection{频分复用FDM}
将信道总频带划分为多个子频带,每个子频带作为一个子信道,每对用户使用一个子信道。









