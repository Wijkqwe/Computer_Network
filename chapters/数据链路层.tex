
\chapter{数据链路层}

\section{功能}

使用物理层提供的\textbf{比特传输}服务;

为网络层提供服务,将网络层的\textbf{IP数据报(分组)}封装称帧,发送给下一结点;


\paragraph{物理链路}
传输介质(0层)+ 物理层(1层)实现相邻结点之间物理链路

\paragraph{逻辑链路}
数据链路层基于物理链路,实现相邻结点之间逻辑上无差错的“数据链路(逻辑链路)”


\subsection{封装成帧(组帧)}

\subsubsection{帧定界}
如何让接收方确定帧的界限


\subsubsection{透明传输}
接收方能够去除帧定界的附加信息


\subsubsection{字符计数法}
每个帧的开头用一个\textbf{定长计数字段}表示帧长。

帧长 = 定长计数字段长度 + 数据部分长度

缺点:任一计数字段出错,会导致后续所有帧无法定界。


\subsubsection{字节填充法}
使用控制字符SOH表示帧开始,EOT表示结束。

若帧的数据部分包含特殊字符(控制字符SOH、EOT,转义字符ESC),则发送方在特殊字符前添加转义字符ESC;接收方做逆处理。


\subsubsection{零比特填充法}
使用特殊比特串`01111110`表示帧开始/结束。

发送方对帧数据部分进行处理,每遇到连续的5个1,填充一个0;接收方做逆处理,每遇到连续的5个1,删除一个0.

HDLC协议、PPP协议使用。


\subsubsection{违规编码法}
违规编码基于曼彻斯特编码IEEE,上跳0下跳1,看中间,中必变。

需要物理层配合。

若周期中间不跳变,则违规。


\subsection{差错控制}


\subsection{可靠传输}



\subsection{流量控制}



\subsection{介质访问控制}



