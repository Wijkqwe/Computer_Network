
\chapter{数据链路层}

\section{功能}

使用物理层提供的\textbf{比特传输}服务;

为网络层提供服务,将网络层的\textbf{IP数据报(分组)}封装称帧,发送给下一结点;


\paragraph{物理链路}
传输介质(0层)+ 物理层(1层)实现相邻结点之间物理链路

\paragraph{逻辑链路}
数据链路层基于物理链路,实现相邻结点之间逻辑上无差错的“数据链路(逻辑链路)”


\subsection{封装成帧(组帧)}

\subsubsection{帧定界}
如何让接收方确定帧的界限


\subsubsection{透明传输}
接收方能够去除帧定界的附加信息


\subsubsection{字符计数法}
每个帧的开头用一个\textbf{定长计数字段}表示帧长。

帧长 = 定长计数字段长度 + 数据部分长度

缺点:任一计数字段出错,会导致后续所有帧无法定界。


\subsubsection{字节填充法}
使用控制字符SOH表示帧开始,EOT表示结束。

若帧的数据部分包含特殊字符(控制字符SOH、EOT,转义字符ESC),则发送方在特殊字符前添加转义字符ESC;接收方做逆处理。


\subsubsection{零比特填充法}
使用特殊比特串`01111110`表示帧开始/结束。

发送方对帧数据部分进行处理,每遇到连续的5个1,填充一个0;接收方做逆处理,每遇到连续的5个1,删除一个0.

HDLC协议、PPP协议使用。


\subsubsection{违规编码法}
违规编码基于曼彻斯特编码IEEE,上跳0下跳1,看中间,中必变。

需要物理层配合。

若周期中间不跳变,则违规。


\subsection{差错控制}

\paragraph{目标}
发现并解决一个帧内部的“位错”。


\subsubsection{检错编码}
接收方\textbf{发现比特错}后丢弃帧,通知发送方重传帧。

\paragraph{奇偶校验码}
整个校验码(有效信息位+校验位)中1的个数。
\begin{itemize}
    \item 奇校验码:1的个数为奇数;
    \item 偶校验码:1的个数为偶数;
\end{itemize}
\subparagraph{偶校验硬件实现}
各信息位进行\textbf{异或}运算,结果为偶校验位。将偶校验的信息位、校验位异或,结果为0则没有错误。

仅能检测处奇数位错误,无纠错能力。


\paragraph{CRC码(循环冗余校验)}
发送接收方约定一个“除数”,K信息位、R校验位作为“被除数”,添加校验位后保证除法余数为0。

\subparagraph{构造}
设生成多项式\(G(x)\),R位数为\(G(x)\)最高次幂,二进制码为多项式系数。信息位左移R位,对二进制码进行模2除法,得到余数即除数,除数有R + 1位。

\subparagraph{纠错}
\begin{itemize}
    \item 可检测所有奇数个错误
    \item 可检测所有双比特错误
    \item 可检测所有小于等于校验位长度的连续错误
    \item 若选择合适多项式,且\(2^R >= K + R + 1\),则可纠正单比特错
\end{itemize}


\subsubsection{纠错编码}

\paragraph{海明校验码}
信息位分组进行偶校验->多个校验位->标注出错位置
\subparagraph{位数}
n位信息位,k位校验位,则\(2^k >= n + k + 1\)。校验位\(P_i\)放到海明位\(2^{i - 1}\)上,信息位按顺序放到其余位置。
\subparagraph{海明位}
\(H_{n + k} ...H_2 H_1\)
\subparagraph{校验位}
\(P_k...P_2P_1\)。以7位海明码为例,\(P_1 = H_3 \oplus H_5 \oplus H_7 = D_1 \oplus D_2 \oplus D_4\)(3低位1,5低位1,6低位0,7低位1)
\subparagraph{格式转换}
海明位\(H_1H_2...H_{n + k}\),校验位\(P_1P_2...P_1\)
\subparagraph{纠错}
\(S_1 = P_1 \oplus D_1 \oplus D_2 \oplus D_4\),0则无错
\subparagraph{纠错能力}
1位
\subparagraph{检错能力}
2位
\subparagraph{全校验位}
\(P_{\text{全}} = S_k...S_1\)\begin{itemize}
    \item 为0且偶校验成功\(\Rightarrow\)无错
    \item 不为0且偶校验失败\(\Rightarrow\)1位错,纠正即可
    \item 不为0且偶校验成功\(\Rightarrow\)2位错,需重传
\end{itemize}


\subsection{可靠传输}



\subsection{流量控制}


\subsection{滑动窗口机制}
可靠传输与流量控制基于滑动窗口机制实现。

接收方通过确认机制控制发送方窗口滑动


\section{三种协议}

\subsection{停止等待协议S-W}
不属于滑动窗口协议。
\begin{itemize}
    \item 滑动窗口:发送窗口\(W_T = 1\),接收窗口\(W_R = 1\)
    \item 确认机制:确认帧ACK\_i,接收方收到i号帧,没有检出错,返回确认帧ACK\_i
    \item 重传机制:超时重传
    \item 帧编号:仅需1bit给帧编号,需要\(W_T + W_R <= 2^n\);用于区别重复帧
\end{itemize}

\subsubsection{异常情况}
\begin{itemize}
    \item 数据帧丢失:超时重传,重置计时器,计时器超时前收到确认帧
    \item 确认帧丢失:超时重传,接收方收到重复帧,丢弃帧并返回重复帧的ACK
    \item 数据帧出错:丢弃帧,等发送方超时重传
\end{itemize}


\subsection{后退N帧协议GBN}
\begin{itemize}
    \item 滑动窗口:发送窗口\(W_T > 1\),接收窗口\(W_R = 1\)
    \item 确认机制:确认帧ACK\_i,接收方收到i号帧,没有检出错,返回确认帧ACK\_i;连续收到多个数据帧,仅传最后一个ACK\_i
    \item 重传机制:超时重传;若未收到ACK\_i,重传i及之后所有帧
    \item 帧编号:至少需nbit给帧编号,需要\(W_T + W_R <= 2^n\);
\end{itemize}

\subsubsection{异常情况}
\begin{itemize}
    \item 数据帧丢失:
    \item 确认帧丢失:
    \item 数据帧出错:
\end{itemize}


\subsection{选择重传协议SR}
\begin{itemize}
    \item 滑动窗口:发送窗口\(W_T > 1\),接收窗口\(W_R > 1\)
    \item 确认机制:\begin{itemize}
        \item 确认帧ACK\_i,接收方收到i号帧,没有检出错,返回确认帧ACK\_i;不支持累计确认
        \item 否认帧:NAK\_i,接收方收到i帧,但出错,丢弃帧并返回NAK\_i
    \end{itemize}
    \item 重传机制:\begin{itemize}
        \item 超时重传;
        \item 请求重传,发送方收到NAK\_i,则重传i
    \end{itemize}
    \item 帧编号:至少需nbit给帧编号,需要\(W_T + W_R <= 2^n\)且\(W_R <= W_T\);
\end{itemize}

\subsubsection{异常情况}
\begin{itemize}
    \item 数据帧丢失:
    \item 确认帧丢失:
    \item 数据帧出错:
\end{itemize}


\subsection{比较}
\begin{itemize}
    \item 滑动窗口协议:S-W不属于滑动窗口协议;GBN、SR属于滑动窗口协议。
    \item \textbf{ARQ协议}(自动重传请求协议):包括S-W、GBN、SR。\begin{itemize}
        \item \textbf{连续ARQ协议}:只包括GBN、SR。
    \end{itemize}
\end{itemize}

\subsubsection{信道利用率}
\(T_D\)数据传输时延,\(RTT\)两倍的单向传输时延,\(T_A\)确认帧传输时延,发送窗口大小\(N\)。
\begin{itemize}
    \item S-W:理想情况下\footnote{没有帧丢失、比特错误等异常情况}\(U = \dfrac{T_D}{T_D + RTT + T_A}\),若\(T_A\)极短,则通常可忽略不计。
    \item GBN、SR:\(U = \dfrac{NT_D}{T_D + RTT + T_A}\);当\(NT_D >= T_D + RTT + T_A\),则U为1(不能大于1);常结合考察帧编号\footnote{\(W_T + W_R <= 2^n\)}。
    \begin{itemize}
        \item 相同nbit给帧编号,GBN\(W_R\)更小,\(W_T\)更大,故信道利用率也更大。
    \end{itemize}
\end{itemize}


\section{介质访问控制MAC}

以太网的MAC协议提供的是\textbf{无连接不可靠服务}。

\begin{itemize}
    \item 信道划分
    \item 随机访问
    \item 轮询访问
\end{itemize}

\subsection{信道划分}

\subsubsection{时分复用TDM}
时间分为等长TDM帧,每个帧分为等长m个时隙,将m个时隙分给m对用户(节点)使用。

缺点:每个节点最多分配到信道总带宽的\(\dfrac{1}{m}\);若某节点暂不发送数据,时隙闲置,信道利用率低。


\subsubsection{统计时分复用STDM}
又称\textbf{异步时分复用}。在TDM基础上,动态分配时隙。

优点:


\subsubsection{频分复用FDM}
将信道总频带划分为多个子频带,每个子频带作为一个子信道,每对用户使用一个子信道。


\subsubsection{波分复用WDM}
即\textbf{光的频分复用}。


\subsubsection{码分复用CDM}
是CDMA的底层原理。

\begin{itemize}
    \item 给各节点分配专属“码片序列”:\begin{itemize}
        \item 码片序列包含m个码片,可看作m维向量;
        \item 各节点m维向量相互正交;
        \item 相互通信的节点知道彼此码片序列;
    \end{itemize}
    \item 发送方发送数据:\begin{itemize}
        \item 发出m个信号值与码片序列相同,表示比特1;
        \item 发出m个信号值与码片序列相反,表示比特0;
    \end{itemize}
    \item 信号在传输过程中叠加:\begin{itemize}
        \item 本质m维向量的加法;
    \end{itemize}
    \item 接收方接收数据:收到叠加信号,从中分离各发送方数据,与发送方码片序列作规格化内积\begin{itemize}
        \item 结果为1,表示比特1;
        \item 结果为-1,表示比特0;
    \end{itemize}
\end{itemize}


\subsection{ALOHA协议}
\begin{itemize}
    \item 纯ALOHA:准备好数据帧就立即发送;
    \item 时隙ALOHA:时隙大小固定=传送一个最长帧所需时间;仅在时隙开始时才发送帧;\begin{itemize}
        \item 发生冲突时,发生冲突的各节点随机推迟若干时隙;
        \item 避免用户发送数据的随机性,降低冲突概率,提高利用率;
    \end{itemize}
\end{itemize}


\subsection{CSMA协议}
载波监听多路访问协议。发送数据前,先监听信道是否空闲,若空闲,则尝试发送。
\begin{itemize}
    \item 1-坚持CSMA协议:坚持表示若信道不空闲则坚持监听\begin{itemize}
        \item 优点:信道利用率高,一旦空闲就被下一信道使用;
        \item 缺点:多个节点准备好时,一旦空闲,多个节点同时发送,冲突概率高;
    \end{itemize}
    \item 非坚持CSMA协议:\begin{itemize}
        \item 优点:多个节点准备好时,若不空闲,则随机推迟一段时间再尝试监听;
        \item 刚空闲时可能不会立即被利用,信道利用率低;
    \end{itemize}
    \item p-坚持CSMA协议:\begin{itemize}
        \item 属于坚持与非坚持的折中,降低冲突的同时提高利用率;
    \end{itemize}
\end{itemize}


\subsection{CSMA/CD协议}
CD表示冲突检测。用于早期以太网(总线型)。

\begin{itemize}
    \item 先听后发,边听边发,冲突停发,随机重发;
    \item 随机重发:截断二进制指数退避算法:随机等待一段时间=r倍\textbf{争用期}\footnote{是一段固定大小的时间,=2*最远单向传播时延},r是随机数\begin{itemize}
        \item 若\(k <= 10\),在\([0, 2^k - 1]\)内随机取一个整数r;
        \item 若\(k > 10\),在\([0, 2^{10} - 1]\)内随机取一个整数r;
    \end{itemize}
    \item \begin{itemize}
        \item 第10次冲突,是随机重发的分水岭;
        \item 第16次冲突,放弃传帧,报告上级(网络层)
    \end{itemize}
\end{itemize}

若争用期内未发生冲突,则不可能再冲突。

CSMA/CD没有ACK机制,若未检测到冲突,则认为帧发送成功。

\paragraph{最短帧长}
= 2 * 最远单向传播时延 * 信道带宽。

若收到帧小于最短帧长,则视为无效帧。

\paragraph{最长帧长}
防止某些节点一直占用信道。

\paragraph{以太网规定}
\begin{itemize}
    \item 最短帧长64B
    \item 最长帧长1518B
\end{itemize}


\subsection{CSMA/CA协议}
CA表示冲突避免。用于IEEE 802.11无线局域网(WiFi)

\begin{itemize}
    \item 发送方:先听后发:忙则退避:\begin{itemize}
        \item 
        \item 
    \end{itemize}
    \item 随机退避原则:
    \item 接收方:停止等待协议:每收到一个正确数据帧发送一个ACK;
\end{itemize}

\paragraph{信道预约机制(可选功能)}
\begin{itemize}
    \item 发送方广播RTS控制帧\footnote{请求发送,包括:源地址、目的地址、通信所需持续时间}(需指明预约时长)
    \item AP(接入点,包括WiFi热点)\footnote{家庭路由器 = 路由器 + 交换机 + AP;}广播CTS控制帧\footnote{允许发送,包括:源地址、目的地址、通信所需持续时间}(需指明预约时长)
    \item 其他无关节点收到CTS后自动禁言一段时间(虚拟载波监听机制),发送方收到CTS后可以发送数据帧;
    \item AP收到数据帧后,进行CRC校验,若无差错则返回ACK帧;
\end{itemize}

\paragraph{帧间间隔IFS}
\begin{itemize}
    \item DIFS:最长IFS,每次帧事务开始时需要等待的时间
    \item SIFS:最短IFS,收到一个帧后需要预留的一段处理时间
    \item PIFS:中等长度IFS,考研可不关注
\end{itemize}


\paragraph{为什么不CSMA/CD?}
\begin{enumerate}
    \item 硬件难以实现边听边发、冲突检测;
    \item 存在隐蔽站问题;
\end{enumerate}


\subsection{令牌传递协议(轮询访问)}
\begin{itemize}
    \item 每次只传一个帧,传完即释放令牌(即生成一个新令牌)。
    \item 无论令牌帧、数据帧只能单向传输。
    \item 需要专门的MAU\footnote{多站接入单元,用于集中控制令牌环网。}实现集中控制;
    \item 适用于负载高的网络(不会冲突)
\end{itemize}


\subsubsection{令牌帧}
帧定界(开始)|令牌号|帧定界(结束)

指明当前获得令牌的节点,只有获得令牌的节点才能发送数据帧;

若获得令牌的节点没有数据要发送,则将令牌传给下一节点;


\subsubsection{数据帧}
帧定界(开始)|令牌号|目的地址|源地址|数据部分|CRC校验码|帧定界(结束)|是否接收

指明数据帧的源地址、目的地址、获得令牌的节点编号、是否已被接收;

从源节点出发,传递一圈后回到源节点;

传递一圈时,目的节点会复制一份数据,并将数据帧标记为已接收;

回到源节点后,若有异常情况,则重发;若无,则将令牌传给下一节点;


\section{局域网}

\subsection{IEEE802与以太网}

\subsubsection{职责}

\subsubsection{层次划分}
数据链路层分为:\begin{itemize}
    \item 逻辑链路控制(LLC)子层:为了兼容各种局域网技术,名存实亡。有线局域网被802.3垄断;无线局域网被802.11垄断
    \item 介质访问控制(MAC)子层:与传输介质有关的部分功能(如组帧、差错检测等)。物理层均使用曼彻斯特编码,物理层在MAC帧前添加前导码8B(7B前同步码,1B定界符)
    \begin{itemize}
        \item DIX...V2标准(市面上常用)
        \item IEEE802.3标准
    \end{itemize}
\end{itemize}


\subsubsection{V2标准MAC帧}
目的地址(6B,48bit)|源地址(6B,48bit)|类型(2B,指明网络层协议)|数据(46~1500,存放IP数据报,短则填充、长则分片,限制最短最长帧长)|FCS(CRC校验码)


\subsubsection{IEEE802.3标准MAC帧}


\subsection{802.11无线局域网}

\subsubsection{分类}
\begin{itemize}
    \item 有固定基础设施:802.11
    \item 无固定基础设施
\end{itemize}


\subsubsection{概念}

\begin{itemize}
    \item 802.11无线局域网是星形拓扑,中心是接入点AP,又称无线接入点WAP。使用\textbf{CSMA/CA协议}实现介质访问控制。
    \item 门户(Portal):将802.11无线局域网接入802.3有限以太网
    \item 基本服务集BSS:一个基站(AP)+多个服务站
    \item 服务集标识符SSID:无线局域网名字,不超过32字节
    \item 基本服务区BSA:一个BSS能覆盖的地理范围
    \item 扩展服务集ESS:将多个AP接入同一分配系统,组成更大服务集
    \item 漫游:一个移动站从一个BSS移动到另一BSS仍可保持通信
\end{itemize}
两个移动站不能直接通信,必须有AP转发。

AP通常可进行帧格式转换,可将在无线链路上传输的802.11帧格式,与在有线链路上传输的以太网帧格式相互转换。
\begin{itemize}
    \item AP与移动点之间无线链路传输
    \item AP之间、AP与路由之间、AP与以太网交换机之间通常有线链路
\end{itemize}


\subsubsection{802.11帧分类}
\begin{enumerate}
    \item 数据帧
    \item 控制帧:ACK、RTS、CTS
    \item 管理帧:探测请求/响应帧
\end{enumerate}


\subsubsection{802.11数据帧格式}



\subsubsection{传输介质适用情况}
\begin{itemize}
    \item 同轴电缆仅半双工
    \item 双绞线:\begin{itemize}
        \item 速率<2.5Gbps,可半双工或全双工
        \item 速率>2.5Gbps,仅全双工
    \end{itemize}
    \item 光纤仅全双工
    \item 默认交换机连接的终端节点都可以全双工
\end{itemize}
不同网段可采用不同标准。


\subsection{三要素}
\begin{enumerate}
    \item 拓扑结构
    \item 传输介质
    \item 介质访问控制方式
\end{enumerate}


\subsection{特点}


\subsection{硬件架构}


\subsection{VLAN}
虚拟局域网。802.1Q
\begin{itemize}
    \item 将一个大型局域网分为多个小型VLAN,\textbf{每个VLAN是一个广播域}。
    \item 需要使用支持VLAN功能的以太网交换机实现。
    \item 每个VLAN对应一个VID
\end{itemize}

\subsubsection{划分方式}
\begin{enumerate}
    \item 基于接口
    \item 基于MAC地址
    \item 基于IP地址,可让VLAN范围跨越路由器,让多个局域网主机组成一个VLAN(需网络层支持)
\end{enumerate}


\subsubsection{802.1Q帧的作用}
主机与交换机之间,传输标准以太网帧;交换机与交换机之间,传输802.1Q帧


\subsubsection{802.1Q帧的结构}
在标准以太网帧的源地址之后插入VLAN标签4B。


\section{广域网}

通信子网使用\textbf{分组交换技术}。

\subsection{点对点协议PPP}
只支持全双工链路。

应满足要求:\begin{itemize}
    \item 
\end{itemize}

无需满足:

三个组成部分:\begin{enumerate}
    \item 一个将IP数据报封装到串行链路的方法;
    \item 链路控制协议LCP;
    \item 网络控制协议NCP;
\end{enumerate}

帧格式:


\subsection{高级数据链路控制HDLC协议}
数据报文可透明传输(0比特插入法)易于硬件实现。

采用全双工通信。

所有帧使用CRC检验,对信息帧进行顺序编号,传输可靠性高。

站:\begin{itemize}
    \item 主站
    \item 从站
    \item 复合站
\end{itemize}

三种数据操作方式:\begin{itemize}
    \item 
\end{itemize}

帧格式


\subsection{比较}
同\begin{itemize}
    \item 都仅全双工
    \item 都透明传输
    \item 都差错检测但不纠错
\end{itemize}

异:


\section{以太网交换机}

有两种交换方式:\textit{直通交换}、\textit{存储转发交换}。

\subsection{特点}
交换机 = 多端口网桥

工作在数据链路层,可根据目的地MAC转发帧


\subsection{自学习功能}
支持即插即用。

\begin{itemize}
    \item 交换表,初始为空,记录[MAC地址, 端口号]的对应关系
    \item 每收到一个帧,将发送方[MAC地址, 端口号]记录到交换表中
    \item \begin{itemize}
        \item 若不知道接收方在哪,把帧广播到除入口外的其他地址
        \item 若知道,则精准转发至端口
    \end{itemize}
    \item 交换表每个表项有“有效时间”,过时作废
\end{itemize}


\subsection{直通交换}
只检测帧的目的MAC地址(6字节,48bit)

优点:转发时延低。

缺点:不适用于需要速率匹配、协议转换、差错检测的线路


\subsection{存储转发交换}
先把帧完整地接收到交换机内部地高速缓存中,进行必要处理后,根据交换表决定从哪个端口转发。

优点:适用于需要速率匹配、协议转换、差错检测的线路

缺点:转发时延高。


