
\chapter{体系结构}

\section{概念}

\subsection{计算机网络}
计算机网络是一个将众多离散的分治的计算机系统,通过通信设备和线路连接起来,由软件实现\textbf{资源共享和信息传递}的系统。

由若干结点和连接结点的链路组成。


\subsection{互连网}
多个计算机通过路由器连接而成。

可以使用任意通信协议。


\subsection{互联网}
由各大ISP(互联网服务提供商)和国际机构组建。


\section{组成}

\subsection{从组成部分看}

\subsubsection{硬件}
\begin{itemize}
    \item 主机(端系统)
    \item 通信设备
    \item 通信链路
\end{itemize}


\subsubsection{软件}


\subsubsection{协议}
由硬件软件共同实现。


\subsection{从工作方式看}

\subsubsection{边缘部分}
直接为用户提供服务(通信、资源共享)

主要由连接到网上的主机和软件组成。


\subsubsection{核心部分部分}
为边缘部分提供服务

由大量网络和连接网络的路由器组成。


\subsection{从逻辑功能看}

\subsubsection{资源子网}


\subsubsection{通信子网}


\section{功能}

\subsection{资源共享}


\subsection{数据通信}


\subsection{分布式处理}


\subsection{提高可靠性}


\subsection{负载均衡}


\subsection{其他}


\section{交换技术}

\subsection{电路交换}
用于电话网络

过程:\begin{enumerate}
    \item 建立连接(尝试占用资源)
    \item 通信(持续占用资源)
    \item 释放连接(归还通信资源)
\end{enumerate}

优点:\begin{itemize}
    \item 数据直送、效率高
    \item 适用于低频次大量交换数据
\end{itemize}

缺点:\begin{itemize}
    \item 
\end{itemize}

\subsubsection{性能}


\subsection{报文交换}
用于电报网络

\subsubsection{性能}


\subsection{分组交换}
用于现代网络

\subsubsection{性能}


\subsection{虚电路交换}
基于分组交换。



\section{分类}

\subsection{按分布范围}

\subsubsection{WAN广域网}

\subsubsection{MAN城域网}
使用以太网技术。

\subsubsection{LAN局域网}
使用以太网技术。

\subsubsection{PAN个域网}


\subsection{按传输技术}


\subsection{按拓扑结构}


\subsection{按使用者}


\subsection{按传输介质}


\section{性能指标}
一条通信线路在逻辑上往往对应一条发送信道和一条接收信道。

\subsection{速率}
又称数据率、比特率、数据传输速率。

单位:bit/s、b/s、bps。

计网默认\(1K = 10^3\);计组操作系统默认\(1K = 2^{10}\)



\subsection{带宽}
某信道所能传送的最高数据率。

节点间通信所能达到的最高速率由带宽、节点性能共同限制。

五类网线:带宽 = 100Mbps


\subsection{吞吐量}
单位时间内通过某网络(信道、接口)的实际数据量。


\subsection{时延}
数据从网络一端到另一端所需时间。

总时延 = 发送时延(传输时延) + 传播时延 、处理时延 + 排队时延

发送时延(传输时延) = \(\dfrac{\text{数据长度bit}}{\text{发送速率bit/s}}\)

传播时延 = \(\dfrac{\text{信道长度m}}{\text{信道中传播速度m/s}}\)


\subsection{时延带宽积}
一条链路中,已从发送端发出但未到接收端的最大比特数。

时延带宽积 = 传播时延 * 带宽

用于设计最短帧长。


\subsection{RTT往返时延}
发送方发完数据到发送方收到接收方发的确认的时间。

RTT = 数据单向传播时延 + 接收方处理时延 + 确认的发送时延 + 确认的传播时延


\subsection{信道利用率}
某个信道百分之多少的时间是有数据通过的

信道利用率 = \(\dfrac{\text{有数据通过的时间}}{\text{有数据通过的时间 + 没有数据通过的时间}}\)

太低浪费资源;太高网络阻塞。


\section{分层结构}

\subsection{设计思想}
在逻辑上划分层次,将功能安排在层次中。


\subsection{三种常见结构}
标号小的在下层。

\subsubsection{OSI}
法律上标准。
\begin{enumerate}
    \item 物理层:集线器
    \item 数据链路层:交换机
    \item 网络层(网际层):路由器
    \item 运输层(传输层)
    \item 会话层
    \item 表示层
    \item 应用层
\end{enumerate}


\subsubsection{TCP/IP}
实际上标准。
\begin{enumerate}
    \item 网络接口层
    \item 网络层(网际层)
    \item 运输层(传输层)
    \item 应用层
\end{enumerate}


\subsubsection{五层模型}
教学用。
\begin{enumerate}
    \item 物理层
    \item 数据链路层
    \item 网络层
    \item 传输层
    \item 应用层
\end{enumerate}


\subsection{网络体系结构}
网络各层及其协议的集合,网络及其构建所完成功能的精确定义(不涉及实现)。

体系结构是抽象的,实现是具体的。

\subsubsection{实体}
第n层中的活动元素(软件+硬件)称为第n层实体。


\subsubsection{协议}
是水平的。

\paragraph{三要素}
\begin{enumerate}
    \item 语法:数据与控制信息的格式
    \item 语义:需发出何种控制信息
    \item 同步(时序):执行各种操作的条件、时序关系
\end{enumerate}


\subsubsection{接口}
又称服务访问点。


\subsubsection{服务}
是垂直的。

下层为相邻上次提供的功能调用。


\subsection{各层之间关系}

每个层次实现一个或多个功能。

上层实体通过接口请求相邻下层的服务。

\subsubsection{PDU协议数据单元}
对等层次之间传送的数据单位


\subsubsection{SDU服务数据单元}
为完成上一层实体所要求的功能所传送的数据单位


\subsubsection{PCI协议控制信息}
控制协议操作的信息。

